
% Default to the notebook output style

    


% Inherit from the specified cell style.




    
\documentclass[11pt]{article}

    
    
    \usepackage[T1]{fontenc}
    % Nicer default font (+ math font) than Computer Modern for most use cases
    \usepackage{mathpazo}

    % Basic figure setup, for now with no caption control since it's done
    % automatically by Pandoc (which extracts ![](path) syntax from Markdown).
    \usepackage{graphicx}
    % We will generate all images so they have a width \maxwidth. This means
    % that they will get their normal width if they fit onto the page, but
    % are scaled down if they would overflow the margins.
    \makeatletter
    \def\maxwidth{\ifdim\Gin@nat@width>\linewidth\linewidth
    \else\Gin@nat@width\fi}
    \makeatother
    \let\Oldincludegraphics\includegraphics
    % Set max figure width to be 80% of text width, for now hardcoded.
    \renewcommand{\includegraphics}[1]{\Oldincludegraphics[width=.8\maxwidth]{#1}}
    % Ensure that by default, figures have no caption (until we provide a
    % proper Figure object with a Caption API and a way to capture that
    % in the conversion process - todo).
    \usepackage{caption}
    \DeclareCaptionLabelFormat{nolabel}{}
    \captionsetup{labelformat=nolabel}

    \usepackage{adjustbox} % Used to constrain images to a maximum size 
    \usepackage{xcolor} % Allow colors to be defined
    \usepackage{enumerate} % Needed for markdown enumerations to work
    \usepackage{geometry} % Used to adjust the document margins
    \usepackage{amsmath} % Equations
    \usepackage{amssymb} % Equations
    \usepackage{textcomp} % defines textquotesingle
    % Hack from http://tex.stackexchange.com/a/47451/13684:
    \AtBeginDocument{%
        \def\PYZsq{\textquotesingle}% Upright quotes in Pygmentized code
    }
    \usepackage{upquote} % Upright quotes for verbatim code
    \usepackage{eurosym} % defines \euro
    \usepackage[mathletters]{ucs} % Extended unicode (utf-8) support
    \usepackage[utf8x]{inputenc} % Allow utf-8 characters in the tex document
    \usepackage{fancyvrb} % verbatim replacement that allows latex
    \usepackage{grffile} % extends the file name processing of package graphics 
                         % to support a larger range 
    % The hyperref package gives us a pdf with properly built
    % internal navigation ('pdf bookmarks' for the table of contents,
    % internal cross-reference links, web links for URLs, etc.)
    \usepackage{hyperref}
    \usepackage{longtable} % longtable support required by pandoc >1.10
    \usepackage{booktabs}  % table support for pandoc > 1.12.2
    \usepackage[inline]{enumitem} % IRkernel/repr support (it uses the enumerate* environment)
    \usepackage[normalem]{ulem} % ulem is needed to support strikethroughs (\sout)
                                % normalem makes italics be italics, not underlines
    

    
    
    % Colors for the hyperref package
    \definecolor{urlcolor}{rgb}{0,.145,.698}
    \definecolor{linkcolor}{rgb}{.71,0.21,0.01}
    \definecolor{citecolor}{rgb}{.12,.54,.11}

    % ANSI colors
    \definecolor{ansi-black}{HTML}{3E424D}
    \definecolor{ansi-black-intense}{HTML}{282C36}
    \definecolor{ansi-red}{HTML}{E75C58}
    \definecolor{ansi-red-intense}{HTML}{B22B31}
    \definecolor{ansi-green}{HTML}{00A250}
    \definecolor{ansi-green-intense}{HTML}{007427}
    \definecolor{ansi-yellow}{HTML}{DDB62B}
    \definecolor{ansi-yellow-intense}{HTML}{B27D12}
    \definecolor{ansi-blue}{HTML}{208FFB}
    \definecolor{ansi-blue-intense}{HTML}{0065CA}
    \definecolor{ansi-magenta}{HTML}{D160C4}
    \definecolor{ansi-magenta-intense}{HTML}{A03196}
    \definecolor{ansi-cyan}{HTML}{60C6C8}
    \definecolor{ansi-cyan-intense}{HTML}{258F8F}
    \definecolor{ansi-white}{HTML}{C5C1B4}
    \definecolor{ansi-white-intense}{HTML}{A1A6B2}

    % commands and environments needed by pandoc snippets
    % extracted from the output of `pandoc -s`
    \providecommand{\tightlist}{%
      \setlength{\itemsep}{0pt}\setlength{\parskip}{0pt}}
    \DefineVerbatimEnvironment{Highlighting}{Verbatim}{commandchars=\\\{\}}
    % Add ',fontsize=\small' for more characters per line
    \newenvironment{Shaded}{}{}
    \newcommand{\KeywordTok}[1]{\textcolor[rgb]{0.00,0.44,0.13}{\textbf{{#1}}}}
    \newcommand{\DataTypeTok}[1]{\textcolor[rgb]{0.56,0.13,0.00}{{#1}}}
    \newcommand{\DecValTok}[1]{\textcolor[rgb]{0.25,0.63,0.44}{{#1}}}
    \newcommand{\BaseNTok}[1]{\textcolor[rgb]{0.25,0.63,0.44}{{#1}}}
    \newcommand{\FloatTok}[1]{\textcolor[rgb]{0.25,0.63,0.44}{{#1}}}
    \newcommand{\CharTok}[1]{\textcolor[rgb]{0.25,0.44,0.63}{{#1}}}
    \newcommand{\StringTok}[1]{\textcolor[rgb]{0.25,0.44,0.63}{{#1}}}
    \newcommand{\CommentTok}[1]{\textcolor[rgb]{0.38,0.63,0.69}{\textit{{#1}}}}
    \newcommand{\OtherTok}[1]{\textcolor[rgb]{0.00,0.44,0.13}{{#1}}}
    \newcommand{\AlertTok}[1]{\textcolor[rgb]{1.00,0.00,0.00}{\textbf{{#1}}}}
    \newcommand{\FunctionTok}[1]{\textcolor[rgb]{0.02,0.16,0.49}{{#1}}}
    \newcommand{\RegionMarkerTok}[1]{{#1}}
    \newcommand{\ErrorTok}[1]{\textcolor[rgb]{1.00,0.00,0.00}{\textbf{{#1}}}}
    \newcommand{\NormalTok}[1]{{#1}}
    
    % Additional commands for more recent versions of Pandoc
    \newcommand{\ConstantTok}[1]{\textcolor[rgb]{0.53,0.00,0.00}{{#1}}}
    \newcommand{\SpecialCharTok}[1]{\textcolor[rgb]{0.25,0.44,0.63}{{#1}}}
    \newcommand{\VerbatimStringTok}[1]{\textcolor[rgb]{0.25,0.44,0.63}{{#1}}}
    \newcommand{\SpecialStringTok}[1]{\textcolor[rgb]{0.73,0.40,0.53}{{#1}}}
    \newcommand{\ImportTok}[1]{{#1}}
    \newcommand{\DocumentationTok}[1]{\textcolor[rgb]{0.73,0.13,0.13}{\textit{{#1}}}}
    \newcommand{\AnnotationTok}[1]{\textcolor[rgb]{0.38,0.63,0.69}{\textbf{\textit{{#1}}}}}
    \newcommand{\CommentVarTok}[1]{\textcolor[rgb]{0.38,0.63,0.69}{\textbf{\textit{{#1}}}}}
    \newcommand{\VariableTok}[1]{\textcolor[rgb]{0.10,0.09,0.49}{{#1}}}
    \newcommand{\ControlFlowTok}[1]{\textcolor[rgb]{0.00,0.44,0.13}{\textbf{{#1}}}}
    \newcommand{\OperatorTok}[1]{\textcolor[rgb]{0.40,0.40,0.40}{{#1}}}
    \newcommand{\BuiltInTok}[1]{{#1}}
    \newcommand{\ExtensionTok}[1]{{#1}}
    \newcommand{\PreprocessorTok}[1]{\textcolor[rgb]{0.74,0.48,0.00}{{#1}}}
    \newcommand{\AttributeTok}[1]{\textcolor[rgb]{0.49,0.56,0.16}{{#1}}}
    \newcommand{\InformationTok}[1]{\textcolor[rgb]{0.38,0.63,0.69}{\textbf{\textit{{#1}}}}}
    \newcommand{\WarningTok}[1]{\textcolor[rgb]{0.38,0.63,0.69}{\textbf{\textit{{#1}}}}}
    
    
    % Define a nice break command that doesn't care if a line doesn't already
    % exist.
    \def\br{\hspace*{\fill} \\* }
    % Math Jax compatability definitions
    \def\gt{>}
    \def\lt{<}
    % Document parameters
    \title{Python R training course - Session 1}
    
    
    

    % Pygments definitions
    
\makeatletter
\def\PY@reset{\let\PY@it=\relax \let\PY@bf=\relax%
    \let\PY@ul=\relax \let\PY@tc=\relax%
    \let\PY@bc=\relax \let\PY@ff=\relax}
\def\PY@tok#1{\csname PY@tok@#1\endcsname}
\def\PY@toks#1+{\ifx\relax#1\empty\else%
    \PY@tok{#1}\expandafter\PY@toks\fi}
\def\PY@do#1{\PY@bc{\PY@tc{\PY@ul{%
    \PY@it{\PY@bf{\PY@ff{#1}}}}}}}
\def\PY#1#2{\PY@reset\PY@toks#1+\relax+\PY@do{#2}}

\expandafter\def\csname PY@tok@cs\endcsname{\let\PY@it=\textit\def\PY@tc##1{\textcolor[rgb]{0.25,0.50,0.50}{##1}}}
\expandafter\def\csname PY@tok@sb\endcsname{\def\PY@tc##1{\textcolor[rgb]{0.73,0.13,0.13}{##1}}}
\expandafter\def\csname PY@tok@o\endcsname{\def\PY@tc##1{\textcolor[rgb]{0.40,0.40,0.40}{##1}}}
\expandafter\def\csname PY@tok@nv\endcsname{\def\PY@tc##1{\textcolor[rgb]{0.10,0.09,0.49}{##1}}}
\expandafter\def\csname PY@tok@dl\endcsname{\def\PY@tc##1{\textcolor[rgb]{0.73,0.13,0.13}{##1}}}
\expandafter\def\csname PY@tok@nb\endcsname{\def\PY@tc##1{\textcolor[rgb]{0.00,0.50,0.00}{##1}}}
\expandafter\def\csname PY@tok@gs\endcsname{\let\PY@bf=\textbf}
\expandafter\def\csname PY@tok@fm\endcsname{\def\PY@tc##1{\textcolor[rgb]{0.00,0.00,1.00}{##1}}}
\expandafter\def\csname PY@tok@sx\endcsname{\def\PY@tc##1{\textcolor[rgb]{0.00,0.50,0.00}{##1}}}
\expandafter\def\csname PY@tok@k\endcsname{\let\PY@bf=\textbf\def\PY@tc##1{\textcolor[rgb]{0.00,0.50,0.00}{##1}}}
\expandafter\def\csname PY@tok@gi\endcsname{\def\PY@tc##1{\textcolor[rgb]{0.00,0.63,0.00}{##1}}}
\expandafter\def\csname PY@tok@se\endcsname{\let\PY@bf=\textbf\def\PY@tc##1{\textcolor[rgb]{0.73,0.40,0.13}{##1}}}
\expandafter\def\csname PY@tok@kp\endcsname{\def\PY@tc##1{\textcolor[rgb]{0.00,0.50,0.00}{##1}}}
\expandafter\def\csname PY@tok@nd\endcsname{\def\PY@tc##1{\textcolor[rgb]{0.67,0.13,1.00}{##1}}}
\expandafter\def\csname PY@tok@vm\endcsname{\def\PY@tc##1{\textcolor[rgb]{0.10,0.09,0.49}{##1}}}
\expandafter\def\csname PY@tok@nf\endcsname{\def\PY@tc##1{\textcolor[rgb]{0.00,0.00,1.00}{##1}}}
\expandafter\def\csname PY@tok@go\endcsname{\def\PY@tc##1{\textcolor[rgb]{0.53,0.53,0.53}{##1}}}
\expandafter\def\csname PY@tok@mf\endcsname{\def\PY@tc##1{\textcolor[rgb]{0.40,0.40,0.40}{##1}}}
\expandafter\def\csname PY@tok@s\endcsname{\def\PY@tc##1{\textcolor[rgb]{0.73,0.13,0.13}{##1}}}
\expandafter\def\csname PY@tok@vi\endcsname{\def\PY@tc##1{\textcolor[rgb]{0.10,0.09,0.49}{##1}}}
\expandafter\def\csname PY@tok@vc\endcsname{\def\PY@tc##1{\textcolor[rgb]{0.10,0.09,0.49}{##1}}}
\expandafter\def\csname PY@tok@gp\endcsname{\let\PY@bf=\textbf\def\PY@tc##1{\textcolor[rgb]{0.00,0.00,0.50}{##1}}}
\expandafter\def\csname PY@tok@c\endcsname{\let\PY@it=\textit\def\PY@tc##1{\textcolor[rgb]{0.25,0.50,0.50}{##1}}}
\expandafter\def\csname PY@tok@si\endcsname{\let\PY@bf=\textbf\def\PY@tc##1{\textcolor[rgb]{0.73,0.40,0.53}{##1}}}
\expandafter\def\csname PY@tok@s2\endcsname{\def\PY@tc##1{\textcolor[rgb]{0.73,0.13,0.13}{##1}}}
\expandafter\def\csname PY@tok@ch\endcsname{\let\PY@it=\textit\def\PY@tc##1{\textcolor[rgb]{0.25,0.50,0.50}{##1}}}
\expandafter\def\csname PY@tok@kn\endcsname{\let\PY@bf=\textbf\def\PY@tc##1{\textcolor[rgb]{0.00,0.50,0.00}{##1}}}
\expandafter\def\csname PY@tok@w\endcsname{\def\PY@tc##1{\textcolor[rgb]{0.73,0.73,0.73}{##1}}}
\expandafter\def\csname PY@tok@cp\endcsname{\def\PY@tc##1{\textcolor[rgb]{0.74,0.48,0.00}{##1}}}
\expandafter\def\csname PY@tok@gt\endcsname{\def\PY@tc##1{\textcolor[rgb]{0.00,0.27,0.87}{##1}}}
\expandafter\def\csname PY@tok@na\endcsname{\def\PY@tc##1{\textcolor[rgb]{0.49,0.56,0.16}{##1}}}
\expandafter\def\csname PY@tok@cm\endcsname{\let\PY@it=\textit\def\PY@tc##1{\textcolor[rgb]{0.25,0.50,0.50}{##1}}}
\expandafter\def\csname PY@tok@mi\endcsname{\def\PY@tc##1{\textcolor[rgb]{0.40,0.40,0.40}{##1}}}
\expandafter\def\csname PY@tok@ni\endcsname{\let\PY@bf=\textbf\def\PY@tc##1{\textcolor[rgb]{0.60,0.60,0.60}{##1}}}
\expandafter\def\csname PY@tok@cpf\endcsname{\let\PY@it=\textit\def\PY@tc##1{\textcolor[rgb]{0.25,0.50,0.50}{##1}}}
\expandafter\def\csname PY@tok@gd\endcsname{\def\PY@tc##1{\textcolor[rgb]{0.63,0.00,0.00}{##1}}}
\expandafter\def\csname PY@tok@nc\endcsname{\let\PY@bf=\textbf\def\PY@tc##1{\textcolor[rgb]{0.00,0.00,1.00}{##1}}}
\expandafter\def\csname PY@tok@c1\endcsname{\let\PY@it=\textit\def\PY@tc##1{\textcolor[rgb]{0.25,0.50,0.50}{##1}}}
\expandafter\def\csname PY@tok@err\endcsname{\def\PY@bc##1{\setlength{\fboxsep}{0pt}\fcolorbox[rgb]{1.00,0.00,0.00}{1,1,1}{\strut ##1}}}
\expandafter\def\csname PY@tok@ow\endcsname{\let\PY@bf=\textbf\def\PY@tc##1{\textcolor[rgb]{0.67,0.13,1.00}{##1}}}
\expandafter\def\csname PY@tok@kr\endcsname{\let\PY@bf=\textbf\def\PY@tc##1{\textcolor[rgb]{0.00,0.50,0.00}{##1}}}
\expandafter\def\csname PY@tok@no\endcsname{\def\PY@tc##1{\textcolor[rgb]{0.53,0.00,0.00}{##1}}}
\expandafter\def\csname PY@tok@m\endcsname{\def\PY@tc##1{\textcolor[rgb]{0.40,0.40,0.40}{##1}}}
\expandafter\def\csname PY@tok@gu\endcsname{\let\PY@bf=\textbf\def\PY@tc##1{\textcolor[rgb]{0.50,0.00,0.50}{##1}}}
\expandafter\def\csname PY@tok@ge\endcsname{\let\PY@it=\textit}
\expandafter\def\csname PY@tok@s1\endcsname{\def\PY@tc##1{\textcolor[rgb]{0.73,0.13,0.13}{##1}}}
\expandafter\def\csname PY@tok@nn\endcsname{\let\PY@bf=\textbf\def\PY@tc##1{\textcolor[rgb]{0.00,0.00,1.00}{##1}}}
\expandafter\def\csname PY@tok@ss\endcsname{\def\PY@tc##1{\textcolor[rgb]{0.10,0.09,0.49}{##1}}}
\expandafter\def\csname PY@tok@nt\endcsname{\let\PY@bf=\textbf\def\PY@tc##1{\textcolor[rgb]{0.00,0.50,0.00}{##1}}}
\expandafter\def\csname PY@tok@sd\endcsname{\let\PY@it=\textit\def\PY@tc##1{\textcolor[rgb]{0.73,0.13,0.13}{##1}}}
\expandafter\def\csname PY@tok@mb\endcsname{\def\PY@tc##1{\textcolor[rgb]{0.40,0.40,0.40}{##1}}}
\expandafter\def\csname PY@tok@gh\endcsname{\let\PY@bf=\textbf\def\PY@tc##1{\textcolor[rgb]{0.00,0.00,0.50}{##1}}}
\expandafter\def\csname PY@tok@sr\endcsname{\def\PY@tc##1{\textcolor[rgb]{0.73,0.40,0.53}{##1}}}
\expandafter\def\csname PY@tok@gr\endcsname{\def\PY@tc##1{\textcolor[rgb]{1.00,0.00,0.00}{##1}}}
\expandafter\def\csname PY@tok@il\endcsname{\def\PY@tc##1{\textcolor[rgb]{0.40,0.40,0.40}{##1}}}
\expandafter\def\csname PY@tok@ne\endcsname{\let\PY@bf=\textbf\def\PY@tc##1{\textcolor[rgb]{0.82,0.25,0.23}{##1}}}
\expandafter\def\csname PY@tok@nl\endcsname{\def\PY@tc##1{\textcolor[rgb]{0.63,0.63,0.00}{##1}}}
\expandafter\def\csname PY@tok@kc\endcsname{\let\PY@bf=\textbf\def\PY@tc##1{\textcolor[rgb]{0.00,0.50,0.00}{##1}}}
\expandafter\def\csname PY@tok@bp\endcsname{\def\PY@tc##1{\textcolor[rgb]{0.00,0.50,0.00}{##1}}}
\expandafter\def\csname PY@tok@kd\endcsname{\let\PY@bf=\textbf\def\PY@tc##1{\textcolor[rgb]{0.00,0.50,0.00}{##1}}}
\expandafter\def\csname PY@tok@sc\endcsname{\def\PY@tc##1{\textcolor[rgb]{0.73,0.13,0.13}{##1}}}
\expandafter\def\csname PY@tok@sa\endcsname{\def\PY@tc##1{\textcolor[rgb]{0.73,0.13,0.13}{##1}}}
\expandafter\def\csname PY@tok@vg\endcsname{\def\PY@tc##1{\textcolor[rgb]{0.10,0.09,0.49}{##1}}}
\expandafter\def\csname PY@tok@kt\endcsname{\def\PY@tc##1{\textcolor[rgb]{0.69,0.00,0.25}{##1}}}
\expandafter\def\csname PY@tok@mo\endcsname{\def\PY@tc##1{\textcolor[rgb]{0.40,0.40,0.40}{##1}}}
\expandafter\def\csname PY@tok@sh\endcsname{\def\PY@tc##1{\textcolor[rgb]{0.73,0.13,0.13}{##1}}}
\expandafter\def\csname PY@tok@mh\endcsname{\def\PY@tc##1{\textcolor[rgb]{0.40,0.40,0.40}{##1}}}

\def\PYZbs{\char`\\}
\def\PYZus{\char`\_}
\def\PYZob{\char`\{}
\def\PYZcb{\char`\}}
\def\PYZca{\char`\^}
\def\PYZam{\char`\&}
\def\PYZlt{\char`\<}
\def\PYZgt{\char`\>}
\def\PYZsh{\char`\#}
\def\PYZpc{\char`\%}
\def\PYZdl{\char`\$}
\def\PYZhy{\char`\-}
\def\PYZsq{\char`\'}
\def\PYZdq{\char`\"}
\def\PYZti{\char`\~}
% for compatibility with earlier versions
\def\PYZat{@}
\def\PYZlb{[}
\def\PYZrb{]}
\makeatother


    % Exact colors from NB
    \definecolor{incolor}{rgb}{0.0, 0.0, 0.5}
    \definecolor{outcolor}{rgb}{0.545, 0.0, 0.0}



    
    % Prevent overflowing lines due to hard-to-break entities
    \sloppy 
    % Setup hyperref package
    \hypersetup{
      breaklinks=true,  % so long urls are correctly broken across lines
      colorlinks=true,
      urlcolor=urlcolor,
      linkcolor=linkcolor,
      citecolor=citecolor,
      }
    % Slightly bigger margins than the latex defaults
    
    \geometry{verbose,tmargin=1in,bmargin=1in,lmargin=1in,rmargin=1in}
    
    

    \begin{document}
    
    
    \maketitle
    
    

    
    Python is nowadays become an active programming language used for
scientic computing, especially it is widely used in data science.
Moreover, Pythonis totally free and prefered by a lot of companies.
Based on this signicantneed, we aim to provide the JVN students the most
important knowledgein Python.

Adapted by Volodymyr Kuleshov and Isaac Caswell from the CS231n

Python tutorial by Justin Johnson
(\url{http://cs231n.github.io/python-numpy-tutorial/}). The theory is
from MIT Opencourseware: A Gentle Introduction to Programming Using
Python
(\href{http://ocw.mit.edu/courses/electrical-engineering-and-computer-science/6-189-a-gentle-introduction-to-programming-using-python-january-iap-2011/readings/}{http://ocw.mit.edu})

\textbf{Lecture list}

\begin{enumerate}
\def\labelenumi{\arabic{enumi}.}
\item
  \textbf{Introduction to Python}

  \begin{itemize}
  \tightlist
  \item
    What and Why is Python: the signicance of Python in computer
    science, data science community and other fields.
  \item
    How to setup and access Python environment (e.g: install, start,
    stop). Students are required to be familiar with using command line
    to carry out some code.
  \item
    Simple Python convention (variables, constants, operations) and
    coding convention (naming variables).
  \item
    Function
  \end{itemize}
\item
  The Python Standard Library, if/loops statements

  \begin{itemize}
  \tightlist
  \item
    Some useful and basic built-in functions (abs, etc.)
  \item
    How to use import to import libraries (ex: math, random). We will
    introduce you several basic libraries that are often used in Python.
  \item
    Introduce Comparison and Boolean Operators (if/else/while/for)
  \item
    Students are also introduced the concept of object-orientation in
    Python.
  \end{itemize}
\item
  Vector/matrix structures, \texttt{numpy} library

  \begin{itemize}
  \tightlist
  \item
    Vector/matrix: structures in Python, some built-in functions,
    Operators for vector/matrix (access, comparation, search)
  \item
    Introduce to module Numpy
  \end{itemize}
\item
  Python data types, File Processing, \texttt{Pandas} library

  \begin{itemize}
  \tightlist
  \item
    Introduce String/List and basic built-in functions and operators for
    String/List.
  \item
    Structure of Python files. Read/create a file.
  \item
    Read csv file using \texttt{Pandas} library
  \end{itemize}
\item
  Functions in Python, Debugging.

  \begin{itemize}
  \tightlist
  \item
    What is functions and how to define functions in Python
  \item
    Passing Arguments to Functions
  \item
    Common errors in Python
  \end{itemize}
\item
  Introduction to R, R for Python programmers.

  \begin{itemize}
  \tightlist
  \item
    From Python to R.
  \item
    Variable, using library, function in R
  \end{itemize}
\item
  Import data, plot data.

  \begin{itemize}
  \tightlist
  \item
    Read csv file, quick summary.
  \item
    Plot data.
  \end{itemize}
\item
  Data Mining in Python/R.
\end{enumerate}

    \section{Introduction to Python}\label{introduction-to-python}

Python is a great general-purpose programming language on its own,
butwith the help of a few popular libraries (\emph{numpy, scipy,
matplotlib}) it becomesa powerful environment for scientific computing.

\subsection{Why we choose Python}\label{why-we-choose-python}

\subsubsection{Popular}\label{popular}

Python is a general-use high-level programming language that bills
itself aspowerful, fast, friendly, open, and easy to learn. Python
``plays well withothers'' and ``runs everywhere''.

Bank of America uses Python to crunch financial data. The
TheoreticalPhysics Division of Los Alamos National Laboratory chose
Python to notonly control simulations, but also analyze and visualize
data. Facebookturns to the Python library Pandas for its data analysis
because it sees thebenefit of using one programming language across
multiple applications.

\subsection{Simple}\label{simple}

Python is easy to use, powerful, and versatile, making it a great choice
for beginners and experts alike. Python's readability makes it a great
first programming language --- it allows you to think like a programmer
and not waste time understanding the mysterious syntax that other
programming languages can require.

\subsection{Write once run every
where}\label{write-once-run-every-where}

Supporting Operate System : Windows, Linux/Unix, MacOS

\subsection{References}\label{references}

\begin{itemize}
\tightlist
\item
  http://www.mastersindatascience.org/data-scientist-skills/python/
\item
  https://www.codeschool.com/blog/2016/01/27/why-python/
\item
  https://www.quora.com/Why-is-Python-a-language-of-choice-for-data-scientists
\item
  http://www.kdnuggets.com/2015/05/r-vs-python-data-science.html
\end{itemize}

    \section{Install Python}\label{install-python}

We highly recommend all of you install Anaconda Python
\url{https://www.continuum.io/downloads}

\begin{figure}
\centering
\includegraphics{figs/session_1/anaconda.png}
\caption{}
\end{figure}

Benefit when using Python Anaconda:

\begin{itemize}
\tightlist
\item
  Easy setup and install a stable environment for Python programming.
\item
  720+ data science packages for interactive data visualizations,
  machine learning, deep learning and Big Data.
\item
  World-class package, dependency and environment management.
\end{itemize}

\section{Basics of Python}\label{basics-of-python}

Python is a high-level, dynamically typed multiparadigm programming
language. Python code is often said to be almost like pseudocode, since
it allows you to express very powerful ideas in very few lines of code
while being very readable. As an example, here is an implementation of
the classic quicksort algorithm in Python

    \begin{Verbatim}[commandchars=\\\{\}]
{\color{incolor}In [{\color{incolor}1}]:} \PY{k}{def} \PY{n+nf}{quicksort}\PY{p}{(}\PY{n}{arr}\PY{p}{)}\PY{p}{:}
            \PY{l+s+sd}{\PYZsq{}\PYZsq{}\PYZsq{}}
        \PY{l+s+sd}{    quicksort function}
        \PY{l+s+sd}{    \PYZsq{}\PYZsq{}\PYZsq{}}
            \PY{k}{if} \PY{n+nb}{len}\PY{p}{(}\PY{n}{arr}\PY{p}{)} \PY{o}{\PYZlt{}}\PY{o}{=} \PY{l+m+mi}{1}\PY{p}{:}
                \PY{k}{return} \PY{n}{arr}
            
            \PY{n}{pivot} \PY{o}{=} \PY{n}{arr}\PY{p}{[}\PY{n+nb}{len}\PY{p}{(}\PY{n}{arr}\PY{p}{)} \PY{o}{/} \PY{l+m+mi}{2}\PY{p}{]}
            \PY{n}{left} \PY{o}{=} \PY{p}{[}\PY{n}{x} \PY{k}{for} \PY{n}{x} \PY{o+ow}{in} \PY{n}{arr} \PY{k}{if} \PY{n}{x} \PY{o}{\PYZlt{}} \PY{n}{pivot}\PY{p}{]}
            \PY{n}{middle} \PY{o}{=} \PY{p}{[}\PY{n}{x} \PY{k}{for} \PY{n}{x} \PY{o+ow}{in} \PY{n}{arr} \PY{k}{if} \PY{n}{x} \PY{o}{==} \PY{n}{pivot}\PY{p}{]}
            \PY{n}{right} \PY{o}{=} \PY{p}{[}\PY{n}{x} \PY{k}{for} \PY{n}{x} \PY{o+ow}{in} \PY{n}{arr} \PY{k}{if} \PY{n}{x} \PY{o}{\PYZgt{}} \PY{n}{pivot}\PY{p}{]}
            
            \PY{k}{return} \PY{n}{quicksort}\PY{p}{(}\PY{n}{left}\PY{p}{)} \PY{o}{+} \PY{n}{middle} \PY{o}{+} \PY{n}{quicksort}\PY{p}{(}\PY{n}{right}\PY{p}{)}
        
        \PY{k}{print} \PY{n}{quicksort}\PY{p}{(}\PY{p}{[}\PY{l+m+mi}{3}\PY{p}{,}\PY{l+m+mi}{6}\PY{p}{,}\PY{l+m+mi}{8}\PY{p}{,}\PY{l+m+mi}{10}\PY{p}{,}\PY{l+m+mi}{1}\PY{p}{,}\PY{l+m+mi}{2}\PY{p}{,}\PY{l+m+mi}{1}\PY{p}{]}\PY{p}{)}
\end{Verbatim}


    \begin{Verbatim}[commandchars=\\\{\}]
[1, 1, 2, 3, 6, 8, 10]

    \end{Verbatim}

    There are currently two different supported versions of Python, 2.7 and
3.4. Somewhat confusingly, Python 3.0 introduced many
backwards-incompatible changes to the language, so code written for 2.7
may not work under 3.4 and vice versa. For this class all code will use
Python 2.7.

You can check your Python version at the command line by running
\texttt{python\ -\/-version}.

    \subsection{Variables, expressions and
statements}\label{variables-expressions-and-statements}

Reference:
http://www.greenteapress.com/thinkpython/thinkCSpy/html/chap02.html

\subsubsection{Values and types}\label{values-and-types}

A value is one of the fundamental things like a letter or a number that
a program manipulates. The values we have seen so far are 2 (the result
when we added \texttt{1\ +\ 1}), and
\texttt{\textquotesingle{}Hello,\ World!\textquotesingle{}}.

    \begin{Verbatim}[commandchars=\\\{\}]
{\color{incolor}In [{\color{incolor}2}]:} \PY{l+m+mi}{1} \PY{o}{+} \PY{l+m+mi}{3}
\end{Verbatim}


\begin{Verbatim}[commandchars=\\\{\}]
{\color{outcolor}Out[{\color{outcolor}2}]:} 4
\end{Verbatim}
            
    \begin{Verbatim}[commandchars=\\\{\}]
{\color{incolor}In [{\color{incolor}3}]:} \PY{l+s+s1}{\PYZsq{}}\PY{l+s+s1}{Hello, World!}\PY{l+s+s1}{\PYZsq{}}
\end{Verbatim}


\begin{Verbatim}[commandchars=\\\{\}]
{\color{outcolor}Out[{\color{outcolor}3}]:} 'Hello, World!'
\end{Verbatim}
            
    These values belong to different \textbf{types}: \texttt{2} is an
\textbf{integer}, and
\texttt{\textquotesingle{}Hello,\ World!\textquotesingle{}} is a
\textbf{string}, so-called because it contains a "string" of letters.

The print statement also works for integers:

    \begin{Verbatim}[commandchars=\\\{\}]
{\color{incolor}In [{\color{incolor}4}]:} \PY{k}{print} \PY{l+m+mi}{2}
\end{Verbatim}


    \begin{Verbatim}[commandchars=\\\{\}]
2

    \end{Verbatim}

    If you are not sure what type a value has, the interpreter can tell you.

    \begin{Verbatim}[commandchars=\\\{\}]
{\color{incolor}In [{\color{incolor}5}]:} \PY{k}{print} \PY{n+nb}{type}\PY{p}{(}\PY{l+s+s1}{\PYZsq{}}\PY{l+s+s1}{17}\PY{l+s+s1}{\PYZsq{}}\PY{p}{)}
        \PY{k}{print} \PY{n+nb}{type}\PY{p}{(}\PY{l+m+mi}{17}\PY{p}{)}
\end{Verbatim}


    \begin{Verbatim}[commandchars=\\\{\}]
<type 'str'>
<type 'int'>

    \end{Verbatim}

    \subsubsection{Variables}\label{variables}

A variable is a name that refers to a value. The \textbf{assignment
statement} creates new variables and gives them values:

    \begin{Verbatim}[commandchars=\\\{\}]
{\color{incolor}In [{\color{incolor}6}]:} \PY{n}{message} \PY{o}{=} \PY{l+s+s2}{\PYZdq{}}\PY{l+s+s2}{JVN}\PY{l+s+s2}{\PYZdq{}}
        \PY{n}{n} \PY{o}{=} \PY{l+m+mi}{10}
\end{Verbatim}


    \begin{Verbatim}[commandchars=\\\{\}]
{\color{incolor}In [{\color{incolor}7}]:} \PY{k}{print} \PY{n}{message}
\end{Verbatim}


    \begin{Verbatim}[commandchars=\\\{\}]
JVN

    \end{Verbatim}

    \begin{Verbatim}[commandchars=\\\{\}]
{\color{incolor}In [{\color{incolor}8}]:} \PY{k}{print} \PY{l+s+s2}{\PYZdq{}}\PY{l+s+s2}{Hi, }\PY{l+s+s2}{\PYZdq{}} \PY{o}{+} \PY{n}{message}
\end{Verbatim}


    \begin{Verbatim}[commandchars=\\\{\}]
Hi, JVN

    \end{Verbatim}

    \textbf{Variable names and keywords}

Programmers generally choose names for their variables that are
meaningful they document what the variable is used for.

Variable names can be arbitrarily long. They can contain both letters
and numbers, but:

\begin{itemize}
\tightlist
\item
  they have to begin with a letter. Although it is legal to use
  uppercase letters, by convention we don't.
\item
  \texttt{JVN} and \texttt{Jvn} are different variables.
\item
  the underscore character (\_) can appear in a name. Such as
  \texttt{my\_name} or \texttt{\_n}
\end{itemize}

If you give a variable an illegal name, you get a syntax error:

    \begin{Verbatim}[commandchars=\\\{\}]
{\color{incolor}In [{\color{incolor}9}]:} \PY{l+m+mi}{76}\PY{n}{trombones} \PY{o}{=} \PY{l+s+s1}{\PYZsq{}}\PY{l+s+s1}{big parade}\PY{l+s+s1}{\PYZsq{}}
\end{Verbatim}


    \begin{Verbatim}[commandchars=\\\{\}]

          File "<ipython-input-9-a5f509298d77>", line 1
        76trombones = 'big parade'
                  \^{}
    SyntaxError: invalid syntax


    \end{Verbatim}

    \begin{Verbatim}[commandchars=\\\{\}]
{\color{incolor}In [{\color{incolor}10}]:} \PY{k}{while} \PY{o}{=} \PY{l+m+mi}{1000000}
\end{Verbatim}


    \begin{Verbatim}[commandchars=\\\{\}]

          File "<ipython-input-10-07be154a2b96>", line 1
        while = 1000000
              \^{}
    SyntaxError: invalid syntax


    \end{Verbatim}

    It turns out that class is one of the Python \textbf{keywords}. Keywords
define the language's rules and structure, and they cannot be used as
variable names.

Python has twenty-nine keywords:

\begin{verbatim}
and       def       exec      if        not       return 
assert    del       finally   import    or        try 
break     elif      for       in        pass      while 
class     else      from      is        print     yield 
continue  except    global    lambda    raise 
\end{verbatim}

    \subsubsection{Statements}\label{statements}

A statement is an instruction that the Python interpreter can execute.
We have seen two kinds of statements: print and assignment.

When you type a statement on the command line, Python executes it and
displays the result, if there is one. The result of a print statement is
a value. Assignment statements don't produce a result.

A script usually contains a sequence of statements. If there is more
than one statement, the results appear one at a time as the statements
execute.

For example, the script

\begin{Shaded}
\begin{Highlighting}[]
\OperatorTok{>>>} \BuiltInTok{print} \DecValTok{1} 
\OperatorTok{>>>}\NormalTok{ x }\OperatorTok{=} \DecValTok{2} 
\OperatorTok{>>>} \BuiltInTok{print}\NormalTok{ x }
\end{Highlighting}
\end{Shaded}

produces the output

\begin{verbatim}
1 
2 
\end{verbatim}

Again, the assignment statement produces no output.

    \subsubsection{Operators and operands}\label{operators-and-operands}

Operators are special symbols that represent computations like addition
and multiplication. The values the operator uses are called operands.

The following are all legal Python expressions whose meaning is more or
less clear:

\begin{Shaded}
\begin{Highlighting}[]
\DecValTok{20}\OperatorTok{+}\DecValTok{32}\NormalTok{   hour}\OperatorTok{-}\DecValTok{1}\NormalTok{   hour}\OperatorTok{*}\DecValTok{60}\OperatorTok{+}\NormalTok{minute   minute}\OperatorTok{/}\DecValTok{60}   \DecValTok{5}\OperatorTok{**}\DecValTok{2}\NormalTok{   (}\DecValTok{5}\OperatorTok{+}\DecValTok{9}\NormalTok{)}\OperatorTok{*}\NormalTok{(}\DecValTok{15}\OperatorTok{-}\DecValTok{7}\NormalTok{) }
\end{Highlighting}
\end{Shaded}

Operators:

\begin{itemize}
\tightlist
\item
  Arithmatic: +, -, *, /, and \% (modulus)
\item
  Comparison: ==, !=, \textless{}, \textgreater{}, \textless{}=,
  \textgreater{}=
\item
  Logical: and, or, not
\item
  Exponentiation: **
\end{itemize}

    \subsection{Functions}\label{functions}

\subsubsection{Function calls}\label{function-calls}

You have already seen one example of a function call:

\begin{Shaded}
\begin{Highlighting}[]
\OperatorTok{>>>} \BuiltInTok{type}\NormalTok{(}\StringTok{"32"}\NormalTok{) }
\OperatorTok{<}\BuiltInTok{type} \StringTok{'str'}\OperatorTok{>} 
\end{Highlighting}
\end{Shaded}

\begin{itemize}
\tightlist
\item
  \textbf{type} is function name, it displays the type of a value or
  variable.
\item
  \textbf{"32"} is \textbf{argument} of the function
\item
  \textbf{\texttt{\textless{}type\ \textquotesingle{}str\textquotesingle{}\textgreater{}}}
  The result is called the return value.
\end{itemize}

Instead of printing the return value, we could assign it to a variable:

\begin{Shaded}
\begin{Highlighting}[]
\OperatorTok{>>>}\NormalTok{ betty }\OperatorTok{=} \BuiltInTok{type}\NormalTok{(}\StringTok{"123456"}\NormalTok{) }
\OperatorTok{>>>} \BuiltInTok{print}\NormalTok{ betty }
\OperatorTok{<}\BuiltInTok{type} \StringTok{'str'}\OperatorTok{>} 
\end{Highlighting}
\end{Shaded}

Some build-in function of python:
\url{https://docs.python.org/2/library/functions.html}

\begin{verbatim}

abs()       divmod()    input()         open()      staticmethod()
all()       enumerate() int()           ord()       str()
any()       eval()      isinstance()    pow()       sum()
bin()       file()      iter()          property()  tuple()
bool()      filter()    len()           range()     type()
dir()       id()        oct()           sorted()            
...
\end{verbatim}

    \begin{Verbatim}[commandchars=\\\{\}]
{\color{incolor}In [{\color{incolor}11}]:} \PY{k}{print} \PY{n+nb}{abs}\PY{p}{(}\PY{o}{\PYZhy{}}\PY{l+m+mi}{10}\PY{p}{)}
\end{Verbatim}


    \begin{Verbatim}[commandchars=\\\{\}]
10

    \end{Verbatim}

    \begin{Verbatim}[commandchars=\\\{\}]
{\color{incolor}In [{\color{incolor}12}]:} \PY{k}{print} \PY{n+nb}{len}\PY{p}{(}\PY{l+s+s2}{\PYZdq{}}\PY{l+s+s2}{JVN Forever}\PY{l+s+s2}{\PYZdq{}}\PY{p}{)}
\end{Verbatim}


    \begin{Verbatim}[commandchars=\\\{\}]
11

    \end{Verbatim}

    \subsubsection{Type conversion \& Type
coercion}\label{type-conversion-type-coercion}

Python provides a collection of built-in functions that convert values
from one type to another. The int function takes any value and converts
it to an integer, if possible, or complains otherwise:

    \begin{Verbatim}[commandchars=\\\{\}]
{\color{incolor}In [{\color{incolor}13}]:} \PY{k}{print} \PY{n+nb}{int}\PY{p}{(}\PY{l+s+s2}{\PYZdq{}}\PY{l+s+s2}{32}\PY{l+s+s2}{\PYZdq{}}\PY{p}{)}
         \PY{k}{print} \PY{n+nb}{int}\PY{p}{(}\PY{l+m+mf}{3.99999}\PY{p}{)} 
\end{Verbatim}


    \begin{Verbatim}[commandchars=\\\{\}]
32
3

    \end{Verbatim}

    \begin{Verbatim}[commandchars=\\\{\}]
{\color{incolor}In [{\color{incolor}14}]:} \PY{k}{print} \PY{n+nb}{int}\PY{p}{(}\PY{l+s+s2}{\PYZdq{}}\PY{l+s+s2}{Hello}\PY{l+s+s2}{\PYZdq{}}\PY{p}{)} 
\end{Verbatim}


    \begin{Verbatim}[commandchars=\\\{\}]

        ---------------------------------------------------------------------------

        ValueError                                Traceback (most recent call last)

        <ipython-input-14-f14c94228214> in <module>()
    ----> 1 print int("Hello")
    

        ValueError: invalid literal for int() with base 10: 'Hello'

    \end{Verbatim}

    \begin{Verbatim}[commandchars=\\\{\}]
{\color{incolor}In [{\color{incolor} }]:} \PY{c+c1}{\PYZsh{} the str function converts to type string:}
        \PY{n+nb}{str}\PY{p}{(}\PY{l+m+mi}{32}\PY{p}{)}
\end{Verbatim}


    \textbf{Type coercion:}

    \begin{Verbatim}[commandchars=\\\{\}]
{\color{incolor}In [{\color{incolor}15}]:} \PY{n}{minute} \PY{o}{=} \PY{l+m+mi}{59}
         \PY{n}{minute} \PY{o}{/} \PY{l+m+mi}{60}
\end{Verbatim}


\begin{Verbatim}[commandchars=\\\{\}]
{\color{outcolor}Out[{\color{outcolor}15}]:} 0
\end{Verbatim}
            
    \begin{Verbatim}[commandchars=\\\{\}]
{\color{incolor}In [{\color{incolor}16}]:} \PY{n}{minute} \PY{o}{=} \PY{l+m+mi}{59}
         \PY{n+nb}{float}\PY{p}{(}\PY{n}{minute}\PY{p}{)} \PY{o}{/} \PY{l+m+mi}{60}
\end{Verbatim}


\begin{Verbatim}[commandchars=\\\{\}]
{\color{outcolor}Out[{\color{outcolor}16}]:} 0.9833333333333333
\end{Verbatim}
            
    \subsubsection{Math functions}\label{math-functions}

Python has a math module that provides most of the familiar mathematical
functions. A \textbf{module} is a file that contains a collection of
related functions grouped together.

Before we can use the functions from a module, we have to import them:

\begin{Shaded}
\begin{Highlighting}[]
\OperatorTok{>>>} \ImportTok{import}\NormalTok{ math }
\end{Highlighting}
\end{Shaded}

    \begin{Verbatim}[commandchars=\\\{\}]
{\color{incolor}In [{\color{incolor}17}]:} \PY{k+kn}{import} \PY{n+nn}{math}
         
         \PY{k}{print} \PY{n}{math}\PY{o}{.}\PY{n}{sin}\PY{p}{(}\PY{l+m+mf}{1.5}\PY{p}{)}
\end{Verbatim}


    \begin{Verbatim}[commandchars=\\\{\}]
0.997494986604

    \end{Verbatim}

    \begin{Verbatim}[commandchars=\\\{\}]
{\color{incolor}In [{\color{incolor}18}]:} \PY{k}{print} \PY{n}{math}\PY{o}{.}\PY{n}{sqrt}\PY{p}{(}\PY{l+m+mi}{2}\PY{p}{)} \PY{o}{/} \PY{l+m+mf}{2.0}
\end{Verbatim}


    \begin{Verbatim}[commandchars=\\\{\}]
0.707106781187

    \end{Verbatim}

    \subsubsection{Adding new functions}\label{adding-new-functions}

Creating new functions to solve your particular problems is one of the
most useful things about a general-purpose programming language.

    \begin{Verbatim}[commandchars=\\\{\}]
{\color{incolor}In [{\color{incolor}19}]:} \PY{k}{def} \PY{n+nf}{foo}\PY{p}{(}\PY{n}{x}\PY{p}{)}\PY{p}{:}
             \PY{n}{x} \PY{o}{=} \PY{n}{x} \PY{o}{+} \PY{l+m+mi}{1}
             \PY{k}{return} \PY{n}{x}
         
         \PY{n}{foo}\PY{p}{(}\PY{l+m+mi}{10}\PY{p}{)}
\end{Verbatim}


\begin{Verbatim}[commandchars=\\\{\}]
{\color{outcolor}Out[{\color{outcolor}19}]:} 11
\end{Verbatim}
            
    Python functions are defined using the \texttt{def} keyword. For
example:

    \begin{Verbatim}[commandchars=\\\{\}]
{\color{incolor}In [{\color{incolor}20}]:} \PY{k}{def} \PY{n+nf}{sign}\PY{p}{(}\PY{n}{x}\PY{p}{)}\PY{p}{:}
             \PY{k}{if} \PY{n}{x} \PY{o}{\PYZgt{}} \PY{l+m+mi}{0}\PY{p}{:}
                 \PY{k}{return} \PY{l+s+s1}{\PYZsq{}}\PY{l+s+s1}{positive}\PY{l+s+s1}{\PYZsq{}}
             \PY{k}{elif} \PY{n}{x} \PY{o}{\PYZlt{}} \PY{l+m+mi}{0}\PY{p}{:}
                 \PY{k}{return} \PY{l+s+s1}{\PYZsq{}}\PY{l+s+s1}{negative}\PY{l+s+s1}{\PYZsq{}}
             \PY{k}{else}\PY{p}{:}
                 \PY{k}{return} \PY{l+s+s1}{\PYZsq{}}\PY{l+s+s1}{zero}\PY{l+s+s1}{\PYZsq{}}
         
         \PY{k}{print} \PY{n}{sign}\PY{p}{(}\PY{o}{\PYZhy{}}\PY{l+m+mi}{1}\PY{p}{)}
         \PY{k}{print} \PY{n}{sign}\PY{p}{(}\PY{l+m+mi}{0}\PY{p}{)}
         \PY{k}{print} \PY{n}{sign}\PY{p}{(}\PY{l+m+mi}{10}\PY{p}{)}
\end{Verbatim}


    \begin{Verbatim}[commandchars=\\\{\}]
negative
zero
positive

    \end{Verbatim}

    \begin{Verbatim}[commandchars=\\\{\}]
{\color{incolor}In [{\color{incolor}21}]:} \PY{k}{def} \PY{n+nf}{add\PYZus{}2}\PY{p}{(}\PY{n}{number}\PY{p}{)}\PY{p}{:}
             \PY{k}{return} \PY{n}{number} \PY{o}{+} \PY{l+m+mi}{2}
         
         \PY{n}{add\PYZus{}2}\PY{p}{(}\PY{l+m+mi}{10}\PY{p}{)}
\end{Verbatim}


\begin{Verbatim}[commandchars=\\\{\}]
{\color{outcolor}Out[{\color{outcolor}21}]:} 12
\end{Verbatim}
            
    We will often define functions to take optional keyword arguments, like
this:

    \begin{Verbatim}[commandchars=\\\{\}]
{\color{incolor}In [{\color{incolor}22}]:} \PY{k}{def} \PY{n+nf}{hello}\PY{p}{(}\PY{n}{name}\PY{p}{,} \PY{n}{loud}\PY{o}{=}\PY{n+nb+bp}{False}\PY{p}{)}\PY{p}{:}
             \PY{k}{if} \PY{n}{loud}\PY{p}{:}
                 \PY{k}{print} \PY{l+s+s1}{\PYZsq{}}\PY{l+s+s1}{HELLO, }\PY{l+s+si}{\PYZpc{}s}\PY{l+s+s1}{\PYZsq{}} \PY{o}{\PYZpc{}} \PY{n}{name}\PY{o}{.}\PY{n}{upper}\PY{p}{(}\PY{p}{)}
             \PY{k}{else}\PY{p}{:}
                 \PY{k}{print} \PY{l+s+s1}{\PYZsq{}}\PY{l+s+s1}{Hello, }\PY{l+s+si}{\PYZpc{}s}\PY{l+s+s1}{!}\PY{l+s+s1}{\PYZsq{}} \PY{o}{\PYZpc{}} \PY{n}{name}
                 
         \PY{n}{hello}\PY{p}{(}\PY{l+s+s1}{\PYZsq{}}\PY{l+s+s1}{Bob}\PY{l+s+s1}{\PYZsq{}}\PY{p}{)}
         \PY{n}{hello}\PY{p}{(}\PY{l+s+s1}{\PYZsq{}}\PY{l+s+s1}{Fred}\PY{l+s+s1}{\PYZsq{}}\PY{p}{,} \PY{n}{loud}\PY{o}{=}\PY{n+nb+bp}{True}\PY{p}{)}
\end{Verbatim}


    \begin{Verbatim}[commandchars=\\\{\}]
Hello, Bob!
HELLO, FRED

    \end{Verbatim}

    Read python document {[}how to use build-in function with the
argument{]}: https://docs.python.org/2/library/functions.html\#len

    \section{References}\label{references}

\begin{itemize}
\tightlist
\item
  Code convention: PEP 8 --- the Style Guide for Python Code -
  \url{http://pep8.org}
\item
  Online Course, code Python in your browser -
  \href{https://www.datacamp.com/courses/intro-to-python-for-data-science}{https://www.datacamp.com}
\item
  Python 2.7.13 documentation - \url{https://docs.python.org/2/}
\item
  Basic Python Exercises -
  \href{https://developers.google.com/edu/python/exercises/basic}{Google
  for Education}
\item
  \href{http://greenteapress.com/wp/learning-with-python/}{{[}Book{]}
  Learning with Python - How to Think Like a Computer Scientist}
\item
  Python Numpy Tutorial -
  \url{http://cs231n.github.io/python-numpy-tutorial/}
\end{itemize}

    \section{Homework exercise}\label{homework-exercise}

\begin{itemize}
\tightlist
\item
  Homework reading: Python Classes
  (\url{https://docs.python.org/2/tutorial/classes.html}), list, set
\item
  Try to use 10 build-in functions.
\end{itemize}

    \subsection{Exercise}\label{exercise}

\subsubsection{a. Odd Or Even}\label{a.-odd-or-even}

Ask the user for a number. Depending on whether the number is even or
odd, print out an appropriate message to the user.

\begin{Shaded}
\begin{Highlighting}[]
\OperatorTok{>>>}\NormalTok{ n }\OperatorTok{=} \BuiltInTok{input}\NormalTok{(}\StringTok{"Input n = "}\NormalTok{)}
\OperatorTok{>>>}\NormalTok{ odd_even(n) }
\end{Highlighting}
\end{Shaded}

Extras:

\begin{itemize}
\item
  If the number is a multiple of 4, print out a different message.

\begin{Shaded}
\begin{Highlighting}[]
\OperatorTok{>>>}\NormalTok{ odd_even(}\DecValTok{7}\NormalTok{)}
  \CommentTok{'odd'}
\OperatorTok{>>>}\NormalTok{ odd_even(}\DecValTok{4}\NormalTok{)}
  \CommentTok{'the number is a multiple of 4'} 
\end{Highlighting}
\end{Shaded}
\item
  Ask the user for two numbers: one number to check (call it num) and
  one number to divide by (check). If check divides evenly into num,
  tell that to the user. If not, print a different appropriate message.

\begin{Shaded}
\begin{Highlighting}[]
\OperatorTok{>>>}\NormalTok{ check(}\DecValTok{10}\NormalTok{,}\DecValTok{2}\NormalTok{)}
    \CommentTok{'10 divides evenly by 2'}

\OperatorTok{>>>}\NormalTok{ check(}\DecValTok{11}\NormalTok{,}\DecValTok{2}\NormalTok{)}
    \CommentTok{'11 does not divide evenly by 2'}
\end{Highlighting}
\end{Shaded}
\end{itemize}

    \begin{Verbatim}[commandchars=\\\{\}]
{\color{incolor}In [{\color{incolor}1}]:} \PY{c+c1}{\PYZsh{} Solution}
        \PY{k}{def} \PY{n+nf}{odd\PYZus{}even}\PY{p}{(}\PY{n}{n}\PY{p}{)}\PY{p}{:}
            \PY{k}{if} \PY{n}{n} \PY{o}{\PYZpc{}} \PY{l+m+mi}{2} \PY{o}{==} \PY{l+m+mi}{0}\PY{p}{:}
                \PY{k}{return} \PY{l+s+s1}{\PYZsq{}}\PY{l+s+s1}{odd}\PY{l+s+s1}{\PYZsq{}}
            \PY{k}{return} \PY{l+s+s1}{\PYZsq{}}\PY{l+s+s1}{even}\PY{l+s+s1}{\PYZsq{}}
        
        \PY{c+c1}{\PYZsh{} Extra 1}
        \PY{k}{def} \PY{n+nf}{odd\PYZus{}even}\PY{p}{(}\PY{n}{n}\PY{p}{)}\PY{p}{:}
            \PY{k}{if} \PY{n}{n} \PY{o}{\PYZpc{}} \PY{l+m+mi}{4}\PY{p}{:}
                \PY{k}{return} \PY{l+s+s1}{\PYZsq{}}\PY{l+s+s1}{the number is a multiple of 4}\PY{l+s+s1}{\PYZsq{}}
            \PY{k}{if} \PY{n}{n} \PY{o}{\PYZpc{}} \PY{l+m+mi}{2} \PY{o}{==} \PY{l+m+mi}{0}\PY{p}{:}
                \PY{k}{return} \PY{l+s+s1}{\PYZsq{}}\PY{l+s+s1}{odd}\PY{l+s+s1}{\PYZsq{}}
            \PY{k}{return} \PY{l+s+s1}{\PYZsq{}}\PY{l+s+s1}{even}\PY{l+s+s1}{\PYZsq{}}
        
        \PY{c+c1}{\PYZsh{} Extra 2}
        \PY{k}{def} \PY{n+nf}{check}\PY{p}{(}\PY{n}{a}\PY{p}{,} \PY{n}{b}\PY{p}{)}\PY{p}{:}
            \PY{k}{if} \PY{n}{a} \PY{o}{\PYZpc{}} \PY{n}{b} \PY{o}{==} \PY{l+m+mi}{0}\PY{p}{:}
                \PY{k}{return} \PY{n}{a}\PY{p}{,} \PY{l+s+s1}{\PYZsq{}}\PY{l+s+s1}{divides evenly by}\PY{l+s+s1}{\PYZsq{}}\PY{p}{,} \PY{n}{b}
            \PY{k}{return} \PY{n}{a}\PY{p}{,} \PY{l+s+s1}{\PYZsq{}}\PY{l+s+s1}{does not divide evenly by}\PY{l+s+s1}{\PYZsq{}}\PY{p}{,} \PY{n}{b}
\end{Verbatim}


    \subsubsection{b. Max Of Three}\label{b.-max-of-three}

Implement a function that takes as input three variables, and returns
the largest of the three. Do this function just in 1 or 2 line.

\begin{Shaded}
\begin{Highlighting}[]
\OperatorTok{>>>}\NormalTok{ max_of_three(}\DecValTok{5}\NormalTok{,}\DecValTok{20}\NormalTok{,}\DecValTok{2}\NormalTok{) }
\DecValTok{20}
\end{Highlighting}
\end{Shaded}

    \begin{Verbatim}[commandchars=\\\{\}]
{\color{incolor}In [{\color{incolor}3}]:} \PY{c+c1}{\PYZsh{} Solution }
        \PY{k}{def} \PY{n+nf}{max\PYZus{}of\PYZus{}three}\PY{p}{(}\PY{n}{a}\PY{p}{,} \PY{n}{b}\PY{p}{,} \PY{n}{c}\PY{p}{)}\PY{p}{:}
            \PY{k}{return} \PY{n+nb}{max}\PY{p}{(}\PY{n+nb}{max}\PY{p}{(}\PY{n}{a}\PY{p}{,} \PY{n}{b}\PY{p}{)}\PY{p}{,} \PY{n}{c}\PY{p}{)}
        
        \PY{n}{max\PYZus{}of\PYZus{}three}\PY{p}{(}\PY{l+m+mi}{1}\PY{p}{,} \PY{l+m+mi}{2}\PY{p}{,} \PY{l+m+mi}{3}\PY{p}{)}
\end{Verbatim}


\begin{Verbatim}[commandchars=\\\{\}]
{\color{outcolor}Out[{\color{outcolor}3}]:} 3
\end{Verbatim}
            
    \subsubsection{\texorpdfstring{c. Max Of Three (without \texttt{max()}
function)}{c. Max Of Three (without max() function)}}\label{c.-max-of-three-without-max-function}

Implement a function that takes as input three variables, and returns
the largest of the three. Do this without using the Python
\texttt{max()} function!

\begin{Shaded}
\begin{Highlighting}[]
\OperatorTok{>>>}\NormalTok{ max_of_three(}\DecValTok{5}\NormalTok{,}\DecValTok{20}\NormalTok{,}\DecValTok{2}\NormalTok{) }
\DecValTok{20}
\end{Highlighting}
\end{Shaded}

    \begin{Verbatim}[commandchars=\\\{\}]
{\color{incolor}In [{\color{incolor}4}]:} \PY{c+c1}{\PYZsh{} Solution}
        \PY{k}{def} \PY{n+nf}{max\PYZus{}of\PYZus{}three}\PY{p}{(}\PY{n}{a}\PY{p}{,} \PY{n}{b}\PY{p}{,} \PY{n}{c}\PY{p}{)}\PY{p}{:}
            \PY{n}{m} \PY{o}{=} \PY{n}{a} \PY{k}{if} \PY{n}{a} \PY{o}{\PYZgt{}} \PY{n}{b} \PY{k}{else} \PY{n}{b}
            \PY{n}{m} \PY{o}{=} \PY{n}{m} \PY{k}{if} \PY{n}{m} \PY{o}{\PYZgt{}} \PY{n}{c} \PY{k}{else} \PY{n}{c}
            \PY{k}{return} \PY{n}{m}
        
        \PY{n}{max\PYZus{}of\PYZus{}three}\PY{p}{(}\PY{l+m+mi}{1}\PY{p}{,}\PY{l+m+mi}{2}\PY{p}{,}\PY{l+m+mi}{3}\PY{p}{)}
\end{Verbatim}


\begin{Verbatim}[commandchars=\\\{\}]
{\color{outcolor}Out[{\color{outcolor}4}]:} 3
\end{Verbatim}
            
    \subsubsection{d. Calculating Area of
Circle}\label{d.-calculating-area-of-circle}

Using \texttt{math}, implement a function that calculating the area of
circle, given by \texttt{radius} arg.

\begin{Shaded}
\begin{Highlighting}[]
\OperatorTok{>>>}\NormalTok{ calculate_area(}\DecValTok{10}\NormalTok{)}
    \FloatTok{314.159265359}
\end{Highlighting}
\end{Shaded}

    \begin{Verbatim}[commandchars=\\\{\}]
{\color{incolor}In [{\color{incolor}5}]:} \PY{c+c1}{\PYZsh{} Solution}
        \PY{k+kn}{import} \PY{n+nn}{math}
        
        \PY{k}{def} \PY{n+nf}{calculate\PYZus{}area}\PY{p}{(}\PY{n}{r}\PY{p}{)}\PY{p}{:}
            \PY{k}{return} \PY{n}{math}\PY{o}{.}\PY{n}{pi} \PY{o}{*} \PY{p}{(}\PY{n}{r}\PY{o}{*}\PY{o}{*}\PY{l+m+mi}{2}\PY{p}{)}
        
        \PY{n}{calculate\PYZus{}area}\PY{p}{(}\PY{l+m+mi}{10}\PY{p}{)}
\end{Verbatim}


\begin{Verbatim}[commandchars=\\\{\}]
{\color{outcolor}Out[{\color{outcolor}5}]:} 314.1592653589793
\end{Verbatim}
            
    \subsubsection{e. Reverse Word Order
Solutions}\label{e.-reverse-word-order-solutions}

Write a program that asks the user for a long string containing multiple
words. Print back to the user the same string, except with the words in
backwards order.

\begin{Shaded}
\begin{Highlighting}[]
\OperatorTok{>>>}\NormalTok{ reverse_str(}\StringTok{"My name is Duyet"}\NormalTok{)}
    \CommentTok{"Duyet is name My"}
\end{Highlighting}
\end{Shaded}

    \begin{Verbatim}[commandchars=\\\{\}]
{\color{incolor}In [{\color{incolor}6}]:} \PY{c+c1}{\PYZsh{} Solution}
        \PY{k}{def} \PY{n+nf}{reverse\PYZus{}str}\PY{p}{(}\PY{n}{s}\PY{p}{)}\PY{p}{:}
            \PY{n}{s} \PY{o}{=} \PY{n}{s}\PY{o}{.}\PY{n}{split}\PY{p}{(}\PY{p}{)}
            \PY{k}{return} \PY{l+s+s1}{\PYZsq{}}\PY{l+s+s1}{ }\PY{l+s+s1}{\PYZsq{}}\PY{o}{.}\PY{n}{join}\PY{p}{(}\PY{n+nb}{reversed}\PY{p}{(}\PY{n}{s}\PY{p}{)}\PY{p}{)}
        
        \PY{n}{reverse\PYZus{}str}\PY{p}{(}\PY{l+s+s2}{\PYZdq{}}\PY{l+s+s2}{Le Van Duyet}\PY{l+s+s2}{\PYZdq{}}\PY{p}{)}
\end{Verbatim}


\begin{Verbatim}[commandchars=\\\{\}]
{\color{outcolor}Out[{\color{outcolor}6}]:} 'Duyet Van Le'
\end{Verbatim}
            
    \subsubsection{f. not\_string}\label{f.-not_string}

Given a string, return a new string where \textbf{"not"} has been added
to the front. However, if the string already begins with "not", return
the string \textbf{unchanged}.

\begin{Shaded}
\begin{Highlighting}[]
\OperatorTok{>>>}\NormalTok{ not_string(}\StringTok{'candy'}\NormalTok{)}
    \CommentTok{'not candy'}
    
\OperatorTok{>>>}\NormalTok{ not_string(}\StringTok{'x'}\NormalTok{)}
    \CommentTok{'not x'}
    
\OperatorTok{>>>}\NormalTok{ not_string(}\StringTok{'not bad'}\NormalTok{)}
    \CommentTok{'not bad'}
\end{Highlighting}
\end{Shaded}

    \begin{Verbatim}[commandchars=\\\{\}]
{\color{incolor}In [{\color{incolor}8}]:} \PY{c+c1}{\PYZsh{} Solution}
        \PY{k}{def} \PY{n+nf}{not\PYZus{}string}\PY{p}{(}\PY{n}{s}\PY{p}{)}\PY{p}{:}
            \PY{n}{s} \PY{o}{=} \PY{n}{s}\PY{o}{.}\PY{n}{strip}\PY{p}{(}\PY{p}{)}
            \PY{k}{if} \PY{n}{s}\PY{p}{[}\PY{p}{:}\PY{l+m+mi}{3}\PY{p}{]} \PY{o}{==} \PY{l+s+s1}{\PYZsq{}}\PY{l+s+s1}{not}\PY{l+s+s1}{\PYZsq{}}\PY{p}{:}
                \PY{k}{return} \PY{n}{s}
            \PY{k}{return} \PY{l+s+s1}{\PYZsq{}}\PY{l+s+s1}{not }\PY{l+s+s1}{\PYZsq{}} \PY{o}{+} \PY{n}{s}
        
        \PY{n}{not\PYZus{}string}\PY{p}{(}\PY{l+s+s2}{\PYZdq{}}\PY{l+s+s2}{not bad}\PY{l+s+s2}{\PYZdq{}}\PY{p}{)}
\end{Verbatim}


\begin{Verbatim}[commandchars=\\\{\}]
{\color{outcolor}Out[{\color{outcolor}8}]:} 'not bad'
\end{Verbatim}
            
    \begin{Verbatim}[commandchars=\\\{\}]
{\color{incolor}In [{\color{incolor}10}]:} \PY{c+c1}{\PYZsh{} Solution 2}
         \PY{k}{def} \PY{n+nf}{not\PYZus{}string}\PY{p}{(}\PY{n}{s}\PY{p}{)}\PY{p}{:}
             \PY{n}{s} \PY{o}{=} \PY{n}{s}\PY{o}{.}\PY{n}{strip}\PY{p}{(}\PY{p}{)}
             \PY{k}{if} \PY{n}{s}\PY{o}{.}\PY{n}{find}\PY{p}{(}\PY{l+s+s1}{\PYZsq{}}\PY{l+s+s1}{not}\PY{l+s+s1}{\PYZsq{}}\PY{p}{)} \PY{o}{==} \PY{l+m+mi}{0}\PY{p}{:}
                 \PY{k}{return} \PY{n}{s}
             \PY{k}{return} \PY{l+s+s1}{\PYZsq{}}\PY{l+s+s1}{not }\PY{l+s+s1}{\PYZsq{}} \PY{o}{+} \PY{n}{s}
         
         \PY{n}{not\PYZus{}string}\PY{p}{(}\PY{l+s+s2}{\PYZdq{}}\PY{l+s+s2}{bad}\PY{l+s+s2}{\PYZdq{}}\PY{p}{)}
\end{Verbatim}


\begin{Verbatim}[commandchars=\\\{\}]
{\color{outcolor}Out[{\color{outcolor}10}]:} 'not bad'
\end{Verbatim}
            
    \subsubsection{g. negative/positive.}\label{g.-negativepositive.}

Given 2 int values, return \textbf{True} if one is \textbf{negative} and
one is \textbf{positive}.

\begin{Shaded}
\begin{Highlighting}[]
\OperatorTok{>>>}\NormalTok{ pos_neg(}\OperatorTok{-}\DecValTok{1}\NormalTok{, }\DecValTok{1}\NormalTok{)}
    \VariableTok{True}
    
\OperatorTok{>>>}\NormalTok{ pos_neg(}\DecValTok{100}\NormalTok{, }\DecValTok{200}\NormalTok{)}
    \VariableTok{False}
\end{Highlighting}
\end{Shaded}

    \begin{Verbatim}[commandchars=\\\{\}]
{\color{incolor}In [{\color{incolor}11}]:} \PY{c+c1}{\PYZsh{} Solution}
         \PY{k}{def} \PY{n+nf}{pos\PYZus{}neg}\PY{p}{(}\PY{n}{a}\PY{p}{,} \PY{n}{b}\PY{p}{)}\PY{p}{:}
             \PY{k}{if} \PY{n}{a} \PY{o}{*} \PY{n}{b} \PY{o}{\PYZlt{}} \PY{l+m+mi}{0}\PY{p}{:}
                 \PY{k}{return} \PY{n+nb+bp}{True}
             \PY{k}{return} \PY{n+nb+bp}{False}
         
         \PY{n}{pos\PYZus{}neg}\PY{p}{(}\PY{o}{\PYZhy{}}\PY{l+m+mi}{1}\PY{p}{,} \PY{l+m+mi}{1}\PY{p}{)}
\end{Verbatim}


\begin{Verbatim}[commandchars=\\\{\}]
{\color{outcolor}Out[{\color{outcolor}11}]:} True
\end{Verbatim}
            
    \subsubsection{h. Ahihi}\label{h.-ahihi}

Given a string and a non-negative int \textbf{n}, return a larger string
that is n copies of the original string.

\begin{Shaded}
\begin{Highlighting}[]
\OperatorTok{>>>}\NormalTok{ string_times(}\StringTok{'Hi'}\NormalTok{, }\DecValTok{2}\NormalTok{)}
    \CommentTok{'HiHi'}
    
\OperatorTok{>>>}\NormalTok{ string_times(}\StringTok{'Hi'}\NormalTok{, }\DecValTok{3}\NormalTok{)}
    \CommentTok{'HiHiHi'}
    
\OperatorTok{>>>}\NormalTok{ string_times(}\StringTok{'Hi'}\NormalTok{, }\DecValTok{1}\NormalTok{)}
    \CommentTok{'Hi'}
\end{Highlighting}
\end{Shaded}

    \begin{Verbatim}[commandchars=\\\{\}]
{\color{incolor}In [{\color{incolor}12}]:} \PY{c+c1}{\PYZsh{} Solution}
         
         \PY{k}{def} \PY{n+nf}{string\PYZus{}times}\PY{p}{(}\PY{n}{s}\PY{p}{,} \PY{n}{n}\PY{p}{)}\PY{p}{:}
             \PY{k}{return} \PY{n}{s} \PY{o}{*} \PY{n}{n}
         
         \PY{n}{string\PYZus{}times}\PY{p}{(}\PY{l+s+s2}{\PYZdq{}}\PY{l+s+s2}{Hi}\PY{l+s+s2}{\PYZdq{}}\PY{p}{,} \PY{l+m+mi}{3}\PY{p}{)}
\end{Verbatim}


\begin{Verbatim}[commandchars=\\\{\}]
{\color{outcolor}Out[{\color{outcolor}12}]:} 'HiHiHi'
\end{Verbatim}
            
    \subsubsection{i. First-half}\label{i.-first-half}

Given a string of even length, return the first half. So the string
\textbf{"WooHoo"} yields \textbf{"Woo"}.

\begin{Shaded}
\begin{Highlighting}[]
\OperatorTok{>>>}\NormalTok{ first_half(}\StringTok{'WooHoo'}\NormalTok{)}
    \CommentTok{'Woo'}
    
\OperatorTok{>>>}\NormalTok{ first_half(}\StringTok{'HelloThere'}\NormalTok{)}
    \CommentTok{'Hello'}
    
\OperatorTok{>>>}\NormalTok{ first_half(}\StringTok{'abcdef'}\NormalTok{)}
    \CommentTok{'abc'}
\end{Highlighting}
\end{Shaded}

    \begin{Verbatim}[commandchars=\\\{\}]
{\color{incolor}In [{\color{incolor}13}]:} \PY{c+c1}{\PYZsh{} Solution}
         \PY{k}{def} \PY{n+nf}{first\PYZus{}half}\PY{p}{(}\PY{n}{s}\PY{p}{)}\PY{p}{:}
             \PY{k}{return} \PY{n}{s}\PY{p}{[}\PY{p}{:}\PY{n+nb}{len}\PY{p}{(}\PY{n}{s}\PY{p}{)} \PY{o}{/} \PY{l+m+mi}{2}\PY{p}{]}
         
         \PY{n}{first\PYZus{}half}\PY{p}{(}\PY{l+s+s1}{\PYZsq{}}\PY{l+s+s1}{HelloThere}\PY{l+s+s1}{\PYZsq{}}\PY{p}{)}
\end{Verbatim}


\begin{Verbatim}[commandchars=\\\{\}]
{\color{outcolor}Out[{\color{outcolor}13}]:} 'Hello'
\end{Verbatim}
            
    \subsubsection{j. Number 6}\label{j.-number-6}

The number 6 is a truly great number. Given two int values, \textbf{a}
and \textbf{b}, return \textbf{True} if either one is 6. Or if
\textbf{their sum} or \textbf{difference} is \textbf{6}.

Note: the function \texttt{abs(num)} computes the absolute value of a
number.

\begin{Shaded}
\begin{Highlighting}[]
\OperatorTok{>>>}\NormalTok{ love6(}\DecValTok{6}\NormalTok{, }\DecValTok{7}\NormalTok{)}
    \VariableTok{True}
\OperatorTok{>>>}\NormalTok{ love6(}\DecValTok{1}\NormalTok{, }\DecValTok{5}\NormalTok{) }\CommentTok{# Sum = 6}
    \VariableTok{True}
\OperatorTok{>>>}\NormalTok{ love6(}\DecValTok{13}\NormalTok{, }\DecValTok{7}\NormalTok{) }\CommentTok{# 13 - 7 = 6}
    \VariableTok{True}
\OperatorTok{>>>}\NormalTok{ love6(}\DecValTok{2}\NormalTok{,}\DecValTok{9}\NormalTok{)}
    \VariableTok{False}
\end{Highlighting}
\end{Shaded}

    \begin{Verbatim}[commandchars=\\\{\}]
{\color{incolor}In [{\color{incolor}17}]:} \PY{c+c1}{\PYZsh{} Solution}
         
         \PY{k}{def} \PY{n+nf}{love6}\PY{p}{(}\PY{n}{a}\PY{p}{,} \PY{n}{b}\PY{p}{)}\PY{p}{:}
             \PY{k}{if} \PY{n}{a} \PY{o}{==} \PY{l+m+mi}{6} \PY{o+ow}{or} \PY{n}{b} \PY{o}{==} \PY{l+m+mi}{6}\PY{p}{:} 
                 \PY{k}{return} \PY{n+nb+bp}{True}
             \PY{k}{if} \PY{n+nb}{abs}\PY{p}{(}\PY{n}{a} \PY{o}{\PYZhy{}} \PY{n}{b}\PY{p}{)} \PY{o}{==} \PY{l+m+mi}{6} \PY{o+ow}{or} \PY{p}{(}\PY{n}{a} \PY{o}{+} \PY{n}{b}\PY{p}{)} \PY{o}{==} \PY{l+m+mi}{6}\PY{p}{:}
                 \PY{k}{return} \PY{n+nb+bp}{True}
             \PY{k}{return} \PY{n+nb+bp}{False}
         
         \PY{k}{print} \PY{n}{love6}\PY{p}{(}\PY{l+m+mi}{6}\PY{p}{,} \PY{l+m+mi}{7}\PY{p}{)}
         \PY{k}{print} \PY{n}{love6}\PY{p}{(}\PY{l+m+mi}{1}\PY{p}{,} \PY{l+m+mi}{5}\PY{p}{)}
         \PY{k}{print} \PY{n}{love6}\PY{p}{(}\PY{l+m+mi}{13}\PY{p}{,} \PY{l+m+mi}{7}\PY{p}{)}
         \PY{k}{print} \PY{n}{love6}\PY{p}{(}\PY{l+m+mi}{2}\PY{p}{,}\PY{l+m+mi}{9}\PY{p}{)}
\end{Verbatim}


    \begin{Verbatim}[commandchars=\\\{\}]
True
True
True
False

    \end{Verbatim}


    % Add a bibliography block to the postdoc
    
    
    
    \end{document}
