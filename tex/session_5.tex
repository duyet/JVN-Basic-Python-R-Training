
% Default to the notebook output style

    


% Inherit from the specified cell style.




    
\documentclass[11pt]{article}

    
    
    \usepackage[T1]{fontenc}
    % Nicer default font (+ math font) than Computer Modern for most use cases
    \usepackage{mathpazo}

    % Basic figure setup, for now with no caption control since it's done
    % automatically by Pandoc (which extracts ![](path) syntax from Markdown).
    \usepackage{graphicx}
    % We will generate all images so they have a width \maxwidth. This means
    % that they will get their normal width if they fit onto the page, but
    % are scaled down if they would overflow the margins.
    \makeatletter
    \def\maxwidth{\ifdim\Gin@nat@width>\linewidth\linewidth
    \else\Gin@nat@width\fi}
    \makeatother
    \let\Oldincludegraphics\includegraphics
    % Set max figure width to be 80% of text width, for now hardcoded.
    \renewcommand{\includegraphics}[1]{\Oldincludegraphics[width=.8\maxwidth]{#1}}
    % Ensure that by default, figures have no caption (until we provide a
    % proper Figure object with a Caption API and a way to capture that
    % in the conversion process - todo).
    \usepackage{caption}
    \DeclareCaptionLabelFormat{nolabel}{}
    \captionsetup{labelformat=nolabel}

    \usepackage{adjustbox} % Used to constrain images to a maximum size 
    \usepackage{xcolor} % Allow colors to be defined
    \usepackage{enumerate} % Needed for markdown enumerations to work
    \usepackage{geometry} % Used to adjust the document margins
    \usepackage{amsmath} % Equations
    \usepackage{amssymb} % Equations
    \usepackage{textcomp} % defines textquotesingle
    % Hack from http://tex.stackexchange.com/a/47451/13684:
    \AtBeginDocument{%
        \def\PYZsq{\textquotesingle}% Upright quotes in Pygmentized code
    }
    \usepackage{upquote} % Upright quotes for verbatim code
    \usepackage{eurosym} % defines \euro
    \usepackage[mathletters]{ucs} % Extended unicode (utf-8) support
    \usepackage[utf8x]{inputenc} % Allow utf-8 characters in the tex document
    \usepackage{fancyvrb} % verbatim replacement that allows latex
    \usepackage{grffile} % extends the file name processing of package graphics 
                         % to support a larger range 
    % The hyperref package gives us a pdf with properly built
    % internal navigation ('pdf bookmarks' for the table of contents,
    % internal cross-reference links, web links for URLs, etc.)
    \usepackage{hyperref}
    \usepackage{longtable} % longtable support required by pandoc >1.10
    \usepackage{booktabs}  % table support for pandoc > 1.12.2
    \usepackage[inline]{enumitem} % IRkernel/repr support (it uses the enumerate* environment)
    \usepackage[normalem]{ulem} % ulem is needed to support strikethroughs (\sout)
                                % normalem makes italics be italics, not underlines
    

    
    
    % Colors for the hyperref package
    \definecolor{urlcolor}{rgb}{0,.145,.698}
    \definecolor{linkcolor}{rgb}{.71,0.21,0.01}
    \definecolor{citecolor}{rgb}{.12,.54,.11}

    % ANSI colors
    \definecolor{ansi-black}{HTML}{3E424D}
    \definecolor{ansi-black-intense}{HTML}{282C36}
    \definecolor{ansi-red}{HTML}{E75C58}
    \definecolor{ansi-red-intense}{HTML}{B22B31}
    \definecolor{ansi-green}{HTML}{00A250}
    \definecolor{ansi-green-intense}{HTML}{007427}
    \definecolor{ansi-yellow}{HTML}{DDB62B}
    \definecolor{ansi-yellow-intense}{HTML}{B27D12}
    \definecolor{ansi-blue}{HTML}{208FFB}
    \definecolor{ansi-blue-intense}{HTML}{0065CA}
    \definecolor{ansi-magenta}{HTML}{D160C4}
    \definecolor{ansi-magenta-intense}{HTML}{A03196}
    \definecolor{ansi-cyan}{HTML}{60C6C8}
    \definecolor{ansi-cyan-intense}{HTML}{258F8F}
    \definecolor{ansi-white}{HTML}{C5C1B4}
    \definecolor{ansi-white-intense}{HTML}{A1A6B2}

    % commands and environments needed by pandoc snippets
    % extracted from the output of `pandoc -s`
    \providecommand{\tightlist}{%
      \setlength{\itemsep}{0pt}\setlength{\parskip}{0pt}}
    \DefineVerbatimEnvironment{Highlighting}{Verbatim}{commandchars=\\\{\}}
    % Add ',fontsize=\small' for more characters per line
    \newenvironment{Shaded}{}{}
    \newcommand{\KeywordTok}[1]{\textcolor[rgb]{0.00,0.44,0.13}{\textbf{{#1}}}}
    \newcommand{\DataTypeTok}[1]{\textcolor[rgb]{0.56,0.13,0.00}{{#1}}}
    \newcommand{\DecValTok}[1]{\textcolor[rgb]{0.25,0.63,0.44}{{#1}}}
    \newcommand{\BaseNTok}[1]{\textcolor[rgb]{0.25,0.63,0.44}{{#1}}}
    \newcommand{\FloatTok}[1]{\textcolor[rgb]{0.25,0.63,0.44}{{#1}}}
    \newcommand{\CharTok}[1]{\textcolor[rgb]{0.25,0.44,0.63}{{#1}}}
    \newcommand{\StringTok}[1]{\textcolor[rgb]{0.25,0.44,0.63}{{#1}}}
    \newcommand{\CommentTok}[1]{\textcolor[rgb]{0.38,0.63,0.69}{\textit{{#1}}}}
    \newcommand{\OtherTok}[1]{\textcolor[rgb]{0.00,0.44,0.13}{{#1}}}
    \newcommand{\AlertTok}[1]{\textcolor[rgb]{1.00,0.00,0.00}{\textbf{{#1}}}}
    \newcommand{\FunctionTok}[1]{\textcolor[rgb]{0.02,0.16,0.49}{{#1}}}
    \newcommand{\RegionMarkerTok}[1]{{#1}}
    \newcommand{\ErrorTok}[1]{\textcolor[rgb]{1.00,0.00,0.00}{\textbf{{#1}}}}
    \newcommand{\NormalTok}[1]{{#1}}
    
    % Additional commands for more recent versions of Pandoc
    \newcommand{\ConstantTok}[1]{\textcolor[rgb]{0.53,0.00,0.00}{{#1}}}
    \newcommand{\SpecialCharTok}[1]{\textcolor[rgb]{0.25,0.44,0.63}{{#1}}}
    \newcommand{\VerbatimStringTok}[1]{\textcolor[rgb]{0.25,0.44,0.63}{{#1}}}
    \newcommand{\SpecialStringTok}[1]{\textcolor[rgb]{0.73,0.40,0.53}{{#1}}}
    \newcommand{\ImportTok}[1]{{#1}}
    \newcommand{\DocumentationTok}[1]{\textcolor[rgb]{0.73,0.13,0.13}{\textit{{#1}}}}
    \newcommand{\AnnotationTok}[1]{\textcolor[rgb]{0.38,0.63,0.69}{\textbf{\textit{{#1}}}}}
    \newcommand{\CommentVarTok}[1]{\textcolor[rgb]{0.38,0.63,0.69}{\textbf{\textit{{#1}}}}}
    \newcommand{\VariableTok}[1]{\textcolor[rgb]{0.10,0.09,0.49}{{#1}}}
    \newcommand{\ControlFlowTok}[1]{\textcolor[rgb]{0.00,0.44,0.13}{\textbf{{#1}}}}
    \newcommand{\OperatorTok}[1]{\textcolor[rgb]{0.40,0.40,0.40}{{#1}}}
    \newcommand{\BuiltInTok}[1]{{#1}}
    \newcommand{\ExtensionTok}[1]{{#1}}
    \newcommand{\PreprocessorTok}[1]{\textcolor[rgb]{0.74,0.48,0.00}{{#1}}}
    \newcommand{\AttributeTok}[1]{\textcolor[rgb]{0.49,0.56,0.16}{{#1}}}
    \newcommand{\InformationTok}[1]{\textcolor[rgb]{0.38,0.63,0.69}{\textbf{\textit{{#1}}}}}
    \newcommand{\WarningTok}[1]{\textcolor[rgb]{0.38,0.63,0.69}{\textbf{\textit{{#1}}}}}
    
    
    % Define a nice break command that doesn't care if a line doesn't already
    % exist.
    \def\br{\hspace*{\fill} \\* }
    % Math Jax compatability definitions
    \def\gt{>}
    \def\lt{<}
    % Document parameters
    \title{Python R training course - Session 5}
    
    
    

    % Pygments definitions
    
\makeatletter
\def\PY@reset{\let\PY@it=\relax \let\PY@bf=\relax%
    \let\PY@ul=\relax \let\PY@tc=\relax%
    \let\PY@bc=\relax \let\PY@ff=\relax}
\def\PY@tok#1{\csname PY@tok@#1\endcsname}
\def\PY@toks#1+{\ifx\relax#1\empty\else%
    \PY@tok{#1}\expandafter\PY@toks\fi}
\def\PY@do#1{\PY@bc{\PY@tc{\PY@ul{%
    \PY@it{\PY@bf{\PY@ff{#1}}}}}}}
\def\PY#1#2{\PY@reset\PY@toks#1+\relax+\PY@do{#2}}

\expandafter\def\csname PY@tok@gd\endcsname{\def\PY@tc##1{\textcolor[rgb]{0.63,0.00,0.00}{##1}}}
\expandafter\def\csname PY@tok@gu\endcsname{\let\PY@bf=\textbf\def\PY@tc##1{\textcolor[rgb]{0.50,0.00,0.50}{##1}}}
\expandafter\def\csname PY@tok@gt\endcsname{\def\PY@tc##1{\textcolor[rgb]{0.00,0.27,0.87}{##1}}}
\expandafter\def\csname PY@tok@gs\endcsname{\let\PY@bf=\textbf}
\expandafter\def\csname PY@tok@gr\endcsname{\def\PY@tc##1{\textcolor[rgb]{1.00,0.00,0.00}{##1}}}
\expandafter\def\csname PY@tok@cm\endcsname{\let\PY@it=\textit\def\PY@tc##1{\textcolor[rgb]{0.25,0.50,0.50}{##1}}}
\expandafter\def\csname PY@tok@vg\endcsname{\def\PY@tc##1{\textcolor[rgb]{0.10,0.09,0.49}{##1}}}
\expandafter\def\csname PY@tok@vi\endcsname{\def\PY@tc##1{\textcolor[rgb]{0.10,0.09,0.49}{##1}}}
\expandafter\def\csname PY@tok@vm\endcsname{\def\PY@tc##1{\textcolor[rgb]{0.10,0.09,0.49}{##1}}}
\expandafter\def\csname PY@tok@mh\endcsname{\def\PY@tc##1{\textcolor[rgb]{0.40,0.40,0.40}{##1}}}
\expandafter\def\csname PY@tok@cs\endcsname{\let\PY@it=\textit\def\PY@tc##1{\textcolor[rgb]{0.25,0.50,0.50}{##1}}}
\expandafter\def\csname PY@tok@ge\endcsname{\let\PY@it=\textit}
\expandafter\def\csname PY@tok@vc\endcsname{\def\PY@tc##1{\textcolor[rgb]{0.10,0.09,0.49}{##1}}}
\expandafter\def\csname PY@tok@il\endcsname{\def\PY@tc##1{\textcolor[rgb]{0.40,0.40,0.40}{##1}}}
\expandafter\def\csname PY@tok@go\endcsname{\def\PY@tc##1{\textcolor[rgb]{0.53,0.53,0.53}{##1}}}
\expandafter\def\csname PY@tok@cp\endcsname{\def\PY@tc##1{\textcolor[rgb]{0.74,0.48,0.00}{##1}}}
\expandafter\def\csname PY@tok@gi\endcsname{\def\PY@tc##1{\textcolor[rgb]{0.00,0.63,0.00}{##1}}}
\expandafter\def\csname PY@tok@gh\endcsname{\let\PY@bf=\textbf\def\PY@tc##1{\textcolor[rgb]{0.00,0.00,0.50}{##1}}}
\expandafter\def\csname PY@tok@ni\endcsname{\let\PY@bf=\textbf\def\PY@tc##1{\textcolor[rgb]{0.60,0.60,0.60}{##1}}}
\expandafter\def\csname PY@tok@nl\endcsname{\def\PY@tc##1{\textcolor[rgb]{0.63,0.63,0.00}{##1}}}
\expandafter\def\csname PY@tok@nn\endcsname{\let\PY@bf=\textbf\def\PY@tc##1{\textcolor[rgb]{0.00,0.00,1.00}{##1}}}
\expandafter\def\csname PY@tok@no\endcsname{\def\PY@tc##1{\textcolor[rgb]{0.53,0.00,0.00}{##1}}}
\expandafter\def\csname PY@tok@na\endcsname{\def\PY@tc##1{\textcolor[rgb]{0.49,0.56,0.16}{##1}}}
\expandafter\def\csname PY@tok@nb\endcsname{\def\PY@tc##1{\textcolor[rgb]{0.00,0.50,0.00}{##1}}}
\expandafter\def\csname PY@tok@nc\endcsname{\let\PY@bf=\textbf\def\PY@tc##1{\textcolor[rgb]{0.00,0.00,1.00}{##1}}}
\expandafter\def\csname PY@tok@nd\endcsname{\def\PY@tc##1{\textcolor[rgb]{0.67,0.13,1.00}{##1}}}
\expandafter\def\csname PY@tok@ne\endcsname{\let\PY@bf=\textbf\def\PY@tc##1{\textcolor[rgb]{0.82,0.25,0.23}{##1}}}
\expandafter\def\csname PY@tok@nf\endcsname{\def\PY@tc##1{\textcolor[rgb]{0.00,0.00,1.00}{##1}}}
\expandafter\def\csname PY@tok@si\endcsname{\let\PY@bf=\textbf\def\PY@tc##1{\textcolor[rgb]{0.73,0.40,0.53}{##1}}}
\expandafter\def\csname PY@tok@s2\endcsname{\def\PY@tc##1{\textcolor[rgb]{0.73,0.13,0.13}{##1}}}
\expandafter\def\csname PY@tok@nt\endcsname{\let\PY@bf=\textbf\def\PY@tc##1{\textcolor[rgb]{0.00,0.50,0.00}{##1}}}
\expandafter\def\csname PY@tok@nv\endcsname{\def\PY@tc##1{\textcolor[rgb]{0.10,0.09,0.49}{##1}}}
\expandafter\def\csname PY@tok@s1\endcsname{\def\PY@tc##1{\textcolor[rgb]{0.73,0.13,0.13}{##1}}}
\expandafter\def\csname PY@tok@dl\endcsname{\def\PY@tc##1{\textcolor[rgb]{0.73,0.13,0.13}{##1}}}
\expandafter\def\csname PY@tok@ch\endcsname{\let\PY@it=\textit\def\PY@tc##1{\textcolor[rgb]{0.25,0.50,0.50}{##1}}}
\expandafter\def\csname PY@tok@m\endcsname{\def\PY@tc##1{\textcolor[rgb]{0.40,0.40,0.40}{##1}}}
\expandafter\def\csname PY@tok@gp\endcsname{\let\PY@bf=\textbf\def\PY@tc##1{\textcolor[rgb]{0.00,0.00,0.50}{##1}}}
\expandafter\def\csname PY@tok@sh\endcsname{\def\PY@tc##1{\textcolor[rgb]{0.73,0.13,0.13}{##1}}}
\expandafter\def\csname PY@tok@ow\endcsname{\let\PY@bf=\textbf\def\PY@tc##1{\textcolor[rgb]{0.67,0.13,1.00}{##1}}}
\expandafter\def\csname PY@tok@sx\endcsname{\def\PY@tc##1{\textcolor[rgb]{0.00,0.50,0.00}{##1}}}
\expandafter\def\csname PY@tok@bp\endcsname{\def\PY@tc##1{\textcolor[rgb]{0.00,0.50,0.00}{##1}}}
\expandafter\def\csname PY@tok@c1\endcsname{\let\PY@it=\textit\def\PY@tc##1{\textcolor[rgb]{0.25,0.50,0.50}{##1}}}
\expandafter\def\csname PY@tok@fm\endcsname{\def\PY@tc##1{\textcolor[rgb]{0.00,0.00,1.00}{##1}}}
\expandafter\def\csname PY@tok@o\endcsname{\def\PY@tc##1{\textcolor[rgb]{0.40,0.40,0.40}{##1}}}
\expandafter\def\csname PY@tok@kc\endcsname{\let\PY@bf=\textbf\def\PY@tc##1{\textcolor[rgb]{0.00,0.50,0.00}{##1}}}
\expandafter\def\csname PY@tok@c\endcsname{\let\PY@it=\textit\def\PY@tc##1{\textcolor[rgb]{0.25,0.50,0.50}{##1}}}
\expandafter\def\csname PY@tok@mf\endcsname{\def\PY@tc##1{\textcolor[rgb]{0.40,0.40,0.40}{##1}}}
\expandafter\def\csname PY@tok@err\endcsname{\def\PY@bc##1{\setlength{\fboxsep}{0pt}\fcolorbox[rgb]{1.00,0.00,0.00}{1,1,1}{\strut ##1}}}
\expandafter\def\csname PY@tok@mb\endcsname{\def\PY@tc##1{\textcolor[rgb]{0.40,0.40,0.40}{##1}}}
\expandafter\def\csname PY@tok@ss\endcsname{\def\PY@tc##1{\textcolor[rgb]{0.10,0.09,0.49}{##1}}}
\expandafter\def\csname PY@tok@sr\endcsname{\def\PY@tc##1{\textcolor[rgb]{0.73,0.40,0.53}{##1}}}
\expandafter\def\csname PY@tok@mo\endcsname{\def\PY@tc##1{\textcolor[rgb]{0.40,0.40,0.40}{##1}}}
\expandafter\def\csname PY@tok@kd\endcsname{\let\PY@bf=\textbf\def\PY@tc##1{\textcolor[rgb]{0.00,0.50,0.00}{##1}}}
\expandafter\def\csname PY@tok@mi\endcsname{\def\PY@tc##1{\textcolor[rgb]{0.40,0.40,0.40}{##1}}}
\expandafter\def\csname PY@tok@kn\endcsname{\let\PY@bf=\textbf\def\PY@tc##1{\textcolor[rgb]{0.00,0.50,0.00}{##1}}}
\expandafter\def\csname PY@tok@cpf\endcsname{\let\PY@it=\textit\def\PY@tc##1{\textcolor[rgb]{0.25,0.50,0.50}{##1}}}
\expandafter\def\csname PY@tok@kr\endcsname{\let\PY@bf=\textbf\def\PY@tc##1{\textcolor[rgb]{0.00,0.50,0.00}{##1}}}
\expandafter\def\csname PY@tok@s\endcsname{\def\PY@tc##1{\textcolor[rgb]{0.73,0.13,0.13}{##1}}}
\expandafter\def\csname PY@tok@kp\endcsname{\def\PY@tc##1{\textcolor[rgb]{0.00,0.50,0.00}{##1}}}
\expandafter\def\csname PY@tok@w\endcsname{\def\PY@tc##1{\textcolor[rgb]{0.73,0.73,0.73}{##1}}}
\expandafter\def\csname PY@tok@kt\endcsname{\def\PY@tc##1{\textcolor[rgb]{0.69,0.00,0.25}{##1}}}
\expandafter\def\csname PY@tok@sc\endcsname{\def\PY@tc##1{\textcolor[rgb]{0.73,0.13,0.13}{##1}}}
\expandafter\def\csname PY@tok@sb\endcsname{\def\PY@tc##1{\textcolor[rgb]{0.73,0.13,0.13}{##1}}}
\expandafter\def\csname PY@tok@sa\endcsname{\def\PY@tc##1{\textcolor[rgb]{0.73,0.13,0.13}{##1}}}
\expandafter\def\csname PY@tok@k\endcsname{\let\PY@bf=\textbf\def\PY@tc##1{\textcolor[rgb]{0.00,0.50,0.00}{##1}}}
\expandafter\def\csname PY@tok@se\endcsname{\let\PY@bf=\textbf\def\PY@tc##1{\textcolor[rgb]{0.73,0.40,0.13}{##1}}}
\expandafter\def\csname PY@tok@sd\endcsname{\let\PY@it=\textit\def\PY@tc##1{\textcolor[rgb]{0.73,0.13,0.13}{##1}}}

\def\PYZbs{\char`\\}
\def\PYZus{\char`\_}
\def\PYZob{\char`\{}
\def\PYZcb{\char`\}}
\def\PYZca{\char`\^}
\def\PYZam{\char`\&}
\def\PYZlt{\char`\<}
\def\PYZgt{\char`\>}
\def\PYZsh{\char`\#}
\def\PYZpc{\char`\%}
\def\PYZdl{\char`\$}
\def\PYZhy{\char`\-}
\def\PYZsq{\char`\'}
\def\PYZdq{\char`\"}
\def\PYZti{\char`\~}
% for compatibility with earlier versions
\def\PYZat{@}
\def\PYZlb{[}
\def\PYZrb{]}
\makeatother


    % Exact colors from NB
    \definecolor{incolor}{rgb}{0.0, 0.0, 0.5}
    \definecolor{outcolor}{rgb}{0.545, 0.0, 0.0}



    
    % Prevent overflowing lines due to hard-to-break entities
    \sloppy 
    % Setup hyperref package
    \hypersetup{
      breaklinks=true,  % so long urls are correctly broken across lines
      colorlinks=true,
      urlcolor=urlcolor,
      linkcolor=linkcolor,
      citecolor=citecolor,
      }
    % Slightly bigger margins than the latex defaults
    
    \geometry{verbose,tmargin=1in,bmargin=1in,lmargin=1in,rmargin=1in}
    
    

    \begin{document}
    
    
    \maketitle
    
    

    
    \textbf{Lecture list}

\begin{enumerate}
\def\labelenumi{\arabic{enumi}.}
\item
  Introduction to Python
\item
  The Python Standard Library, if/loops statements
\item
  Vector/matrix structures, \texttt{numpy} library
\item
  Python data types, File Processing, \texttt{Pandas} library
\item
  \textbf{Functions in Python, Debugging.}

  \begin{itemize}
  \tightlist
  \item
    Lambda function.
  \item
    Recursion.
  \item
    Common errors in Python.
  \end{itemize}
\item
  Introduction to R, R for Python programmers.
\item
  Import data, plot data.
\item
  Data Mining in Python/R.
\end{enumerate}

    \section{Lambda function}\label{lambda-function}

\subsection{Syntax of Lambda Function}\label{syntax-of-lambda-function}

In Python, anonymous function (lambda function) is a function that is
defined without a name.

While normal functions are defined using the \textbf{def} keyword, in
Python anonymous functions are defined using the \textbf{lambda}
keyword. Hence, anonymous functions are also called lambda functions.

Syntax:

\begin{Shaded}
\begin{Highlighting}[]
\KeywordTok{lambda}\NormalTok{ arguments: expression}
\end{Highlighting}
\end{Shaded}

Example:

\begin{Shaded}
\begin{Highlighting}[]
\OperatorTok{>>>}\NormalTok{ S }\OperatorTok{=} \KeywordTok{lambda}\NormalTok{ a, b: a }\OperatorTok{+}\NormalTok{ b}
\OperatorTok{>>>}\NormalTok{ S(}\DecValTok{5}\NormalTok{, }\DecValTok{7}\NormalTok{)}
    \DecValTok{12}
\end{Highlighting}
\end{Shaded}

\begin{Shaded}
\begin{Highlighting}[]
\OperatorTok{>>>}\NormalTok{ double }\OperatorTok{=} \KeywordTok{lambda}\NormalTok{ x: x }\OperatorTok{*} \DecValTok{2}
\OperatorTok{>>>}\NormalTok{ double(}\DecValTok{5}\NormalTok{)}
    \DecValTok{10}
\end{Highlighting}
\end{Shaded}

Lambda functions can have any number of arguments but only one
expression. The expression is evaluated and returned. Lambda functions
can be used wherever function objects are required.

In the above program, \textbf{lambda x: x * 2} is the
\texttt{lambda\ function}. Here \textbf{x} is the argument and \textbf{x
* 2} is the expression that gets evaluated and returned.

This function has no name. It returns a function object which is
assigned to the identifier \textbf{double}. We can now call it as a
normal function. The statement

\begin{Shaded}
\begin{Highlighting}[]
\NormalTok{double }\OperatorTok{=} \KeywordTok{lambda}\NormalTok{ x: x }\OperatorTok{*} \DecValTok{2}
\end{Highlighting}
\end{Shaded}

is nearly the same as

\begin{Shaded}
\begin{Highlighting}[]
\KeywordTok{def}\NormalTok{ double(x):}
    \ControlFlowTok{return}\NormalTok{ x }\OperatorTok{*} \DecValTok{2}
\end{Highlighting}
\end{Shaded}

    \subsection{Use of Lambda Function}\label{use-of-lambda-function}

We use lambda functions when we require a nameless function for a short
period of time.

In Python, we generally use it as an argument to a higher-order function
(a function that takes in other functions as
\href{https://www.programiz.com/python-programming/function-argument}{arguments}).
Lambda functions are used along with built-in functions like
\texttt{filter()}, \texttt{map()} etc.

\textbf{Example use with filter()}

The \textbf{filter()} function in Python takes in a function and a list
as arguments.

The function is called with all the items in the list and a new list is
returned which contains items for which the function evaluats to
\texttt{True}.

Here is an example use of filter() function to filter out only even
numbers from a list.

\begin{Shaded}
\begin{Highlighting}[]
\OperatorTok{>>>} \KeywordTok{def}\NormalTok{ is_odd(x):}
\OperatorTok{>>>}     \ControlFlowTok{return}\NormalTok{ x }\OperatorTok{%} \DecValTok{2} \OperatorTok{==} \DecValTok{0}
\OperatorTok{>>>}
\OperatorTok{>>>}\NormalTok{ my_list }\OperatorTok{=}\NormalTok{ [}\DecValTok{1}\NormalTok{, }\DecValTok{5}\NormalTok{, }\DecValTok{4}\NormalTok{, }\DecValTok{6}\NormalTok{, }\DecValTok{8}\NormalTok{, }\DecValTok{11}\NormalTok{, }\DecValTok{3}\NormalTok{, }\DecValTok{12}\NormalTok{]}
\OperatorTok{>>>}\NormalTok{ new_list }\OperatorTok{=} \BuiltInTok{list}\NormalTok{(}\BuiltInTok{filter}\NormalTok{(is_odd, my_list))}
\NormalTok{    [}\DecValTok{4}\NormalTok{, }\DecValTok{6}\NormalTok{, }\DecValTok{8}\NormalTok{, }\DecValTok{12}\NormalTok{]}
\end{Highlighting}
\end{Shaded}

Instead, we can use lambda function:

\begin{Shaded}
\begin{Highlighting}[]
\OperatorTok{>>>}\NormalTok{ my_list }\OperatorTok{=}\NormalTok{ [}\DecValTok{1}\NormalTok{, }\DecValTok{5}\NormalTok{, }\DecValTok{4}\NormalTok{, }\DecValTok{6}\NormalTok{, }\DecValTok{8}\NormalTok{, }\DecValTok{11}\NormalTok{, }\DecValTok{3}\NormalTok{, }\DecValTok{12}\NormalTok{]}
\OperatorTok{>>>}\NormalTok{ new_list }\OperatorTok{=} \BuiltInTok{list}\NormalTok{(}\BuiltInTok{filter}\NormalTok{(}\KeywordTok{lambda}\NormalTok{ x: x }\OperatorTok{%} \DecValTok{2} \OperatorTok{==} \DecValTok{0}\NormalTok{, my_list))}
\NormalTok{    [}\DecValTok{4}\NormalTok{, }\DecValTok{6}\NormalTok{, }\DecValTok{8}\NormalTok{, }\DecValTok{12}\NormalTok{]}
\end{Highlighting}
\end{Shaded}

    \textbf{Example use with map()}

The \textbf{map()} function in Python takes in a function and a list.

The function is called with all the items in the list and a new list is
returned which contains items returned by that function for each item.

Here is an example use of \textbf{map()} function to double all the
items in a list.

\begin{Shaded}
\begin{Highlighting}[]
\OperatorTok{>>>}\NormalTok{ my_list }\OperatorTok{=}\NormalTok{ [}\DecValTok{1}\NormalTok{, }\DecValTok{5}\NormalTok{, }\DecValTok{4}\NormalTok{, }\DecValTok{6}\NormalTok{, }\DecValTok{8}\NormalTok{, }\DecValTok{11}\NormalTok{, }\DecValTok{3}\NormalTok{, }\DecValTok{12}\NormalTok{]}
\OperatorTok{>>>}\NormalTok{ new_list }\OperatorTok{=} \BuiltInTok{list}\NormalTok{(}\BuiltInTok{map}\NormalTok{(}\KeywordTok{lambda}\NormalTok{ x: x }\OperatorTok{**} \DecValTok{2}\NormalTok{ , my_list))}
\NormalTok{    [}\DecValTok{1}\NormalTok{, }\DecValTok{25}\NormalTok{, }\DecValTok{16}\NormalTok{, }\DecValTok{36}\NormalTok{, }\DecValTok{64}\NormalTok{, }\DecValTok{121}\NormalTok{, }\DecValTok{9}\NormalTok{, }\DecValTok{144}\NormalTok{]}
\end{Highlighting}
\end{Shaded}

    \textbf{Example use with Pandas DataFrames}

For example, we have the dataframe:

    \begin{Verbatim}[commandchars=\\\{\}]
{\color{incolor}In [{\color{incolor}10}]:} \PY{k+kn}{import} \PY{n+nn}{pandas} \PY{k+kn}{as} \PY{n+nn}{pd}
         \PY{k+kn}{from} \PY{n+nn}{StringIO} \PY{k+kn}{import} \PY{n}{StringIO}
         
         \PY{n}{df} \PY{o}{=} \PY{n}{pd}\PY{o}{.}\PY{n}{read\PYZus{}csv}\PY{p}{(}\PY{n}{StringIO}\PY{p}{(}\PY{l+s+s2}{\PYZdq{}}\PY{l+s+s2}{col1,col2,col3}\PY{l+s+se}{\PYZbs{}n}\PY{l+s+s2}{a,b,1}\PY{l+s+se}{\PYZbs{}n}\PY{l+s+s2}{a,b,2}\PY{l+s+se}{\PYZbs{}n}\PY{l+s+s2}{c,d,3}\PY{l+s+s2}{\PYZdq{}}\PY{p}{)}\PY{p}{)}
         \PY{n}{df}
\end{Verbatim}

            \begin{Verbatim}[commandchars=\\\{\}]
{\color{outcolor}Out[{\color{outcolor}10}]:}   col1 col2  col3
         0    a    b     1
         1    a    b     2
         2    c    d     3
\end{Verbatim}
        
    Now, we'll start new \textbf{col3\_double} from \textbf{col3} by apply
the double function.

    \begin{Verbatim}[commandchars=\\\{\}]
{\color{incolor}In [{\color{incolor}12}]:} \PY{k}{def} \PY{n+nf}{double}\PY{p}{(}\PY{n}{x}\PY{p}{)}\PY{p}{:}
             \PY{k}{return} \PY{n}{x} \PY{o}{*} \PY{l+m+mi}{2}
         
         \PY{n}{df}\PY{p}{[}\PY{l+s+s1}{\PYZsq{}}\PY{l+s+s1}{col3\PYZus{}double}\PY{l+s+s1}{\PYZsq{}}\PY{p}{]} \PY{o}{=} \PY{n}{df}\PY{o}{.}\PY{n}{col3}\PY{o}{.}\PY{n}{apply}\PY{p}{(}\PY{n}{double}\PY{p}{)}
         \PY{n}{df}
\end{Verbatim}

            \begin{Verbatim}[commandchars=\\\{\}]
{\color{outcolor}Out[{\color{outcolor}12}]:}   col1 col2  col3  col3\_double
         0    a    b     1            2
         1    a    b     2            4
         2    c    d     3            6
\end{Verbatim}
        
    More quickly with \textbf{lambda function} here:

    \begin{Verbatim}[commandchars=\\\{\}]
{\color{incolor}In [{\color{incolor}14}]:} \PY{n}{df}\PY{p}{[}\PY{l+s+s1}{\PYZsq{}}\PY{l+s+s1}{col3\PYZus{}double}\PY{l+s+s1}{\PYZsq{}}\PY{p}{]} \PY{o}{=} \PY{n}{df}\PY{o}{.}\PY{n}{col3}\PY{o}{.}\PY{n}{apply}\PY{p}{(}\PY{k}{lambda} \PY{n}{x}\PY{p}{:} \PY{n}{x} \PY{o}{*} \PY{l+m+mi}{2}\PY{p}{)}
         \PY{n}{df}
\end{Verbatim}

            \begin{Verbatim}[commandchars=\\\{\}]
{\color{outcolor}Out[{\color{outcolor}14}]:}   col1 col2  col3  col3\_double
         0    a    b     1            2
         1    a    b     2            4
         2    c    d     3            6
\end{Verbatim}
        
    \begin{Verbatim}[commandchars=\\\{\}]
{\color{incolor}In [{\color{incolor}18}]:} \PY{n}{df}\PY{o}{.}\PY{n}{col3}\PY{o}{.}\PY{n}{map}\PY{p}{(}\PY{k}{lambda} \PY{n}{t}\PY{p}{:} \PY{n}{t} \PY{o}{\PYZgt{}} \PY{l+m+mi}{2}\PY{p}{)}
\end{Verbatim}

            \begin{Verbatim}[commandchars=\\\{\}]
{\color{outcolor}Out[{\color{outcolor}18}]:} 0    False
         1    False
         2     True
         Name: col3, dtype: bool
\end{Verbatim}
        
    \section{Python Recursion}\label{python-recursion}

Recursion is the process of defining something in terms of itself.

We know that in Python, a function can call other functions. It is even
possible for the function to call itself. These type of construct are
termed as recursive functions.

Following is an example of recursive function to find the factorial of
an integer.

Factorial of a number is the product of all the integers from 1 to that
number. For example, the factorial of 6 (denoted as 6!) is
\texttt{1*2*3*4*5*6\ =\ 720}.

    \begin{Verbatim}[commandchars=\\\{\}]
{\color{incolor}In [{\color{incolor}20}]:} \PY{k}{def} \PY{n+nf}{calc\PYZus{}factorial}\PY{p}{(}\PY{n}{x}\PY{p}{)}\PY{p}{:}
             \PY{l+s+sd}{\PYZdq{}\PYZdq{}\PYZdq{}This is a recursive function}
         \PY{l+s+sd}{    to find the factorial of an integer\PYZdq{}\PYZdq{}\PYZdq{}}
         
             \PY{k}{if} \PY{n}{x} \PY{o}{==} \PY{l+m+mi}{1}\PY{p}{:}
                 \PY{k}{return} \PY{l+m+mi}{1}
             \PY{k}{else}\PY{p}{:}
                 \PY{k}{return} \PY{p}{(}\PY{n}{x} \PY{o}{*} \PY{n}{calc\PYZus{}factorial}\PY{p}{(}\PY{n}{x}\PY{o}{\PYZhy{}}\PY{l+m+mi}{1}\PY{p}{)}\PY{p}{)}
         
         \PY{n}{num} \PY{o}{=} \PY{l+m+mi}{4}
         \PY{k}{print} \PY{l+s+s2}{\PYZdq{}}\PY{l+s+s2}{The factorial of}\PY{l+s+s2}{\PYZdq{}}\PY{p}{,} \PY{n}{num}\PY{p}{,} \PY{l+s+s2}{\PYZdq{}}\PY{l+s+s2}{is}\PY{l+s+s2}{\PYZdq{}}\PY{p}{,} \PY{n}{calc\PYZus{}factorial}\PY{p}{(}\PY{n}{num}\PY{p}{)}
\end{Verbatim}

    \begin{Verbatim}[commandchars=\\\{\}]
The factorial of 4 is 24

    \end{Verbatim}

    In the above example, \textbf{calc\_factorial()} is a recursive
functions as it calls itself.

When we call this function with a positive integer, it will recursively
call itself by decreasing the number.

Each function call multiples the number with the factorial of number 1
until the number is equal to one. This recursive call can be explained
in the following steps.

\begin{Shaded}
\begin{Highlighting}[]
\NormalTok{calc_factorial(}\DecValTok{4}\NormalTok{)              }\CommentTok{# 1st call with 4}
\DecValTok{4} \OperatorTok{*}\NormalTok{ calc_factorial(}\DecValTok{3}\NormalTok{)          }\CommentTok{# 2nd call with 3}
\DecValTok{4} \OperatorTok{*} \DecValTok{3} \OperatorTok{*}\NormalTok{ calc_factorial(}\DecValTok{2}\NormalTok{)      }\CommentTok{# 3rd call with 2}
\DecValTok{4} \OperatorTok{*} \DecValTok{3} \OperatorTok{*} \DecValTok{2} \OperatorTok{*}\NormalTok{ calc_factorial(}\DecValTok{1}\NormalTok{)  }\CommentTok{# 4th call with 1}
\DecValTok{4} \OperatorTok{*} \DecValTok{3} \OperatorTok{*} \DecValTok{2} \OperatorTok{*} \DecValTok{1}                  \CommentTok{# return from 4th call as number=1}
\DecValTok{4} \OperatorTok{*} \DecValTok{3} \OperatorTok{*} \DecValTok{2}                      \CommentTok{# return from 3rd call}
\DecValTok{4} \OperatorTok{*} \DecValTok{6}                          \CommentTok{# return from 2nd call}
\DecValTok{24}                             \CommentTok{# return from 1st call}
\end{Highlighting}
\end{Shaded}

    Our recursion ends when the number reduces to 1. This is called the base
condition.

Every recursive function must have a base condition that stops the
recursion or else the function calls itself infinitely.

\begin{itemize}
\tightlist
\item
  Advantages of recursion

  \begin{itemize}
  \tightlist
  \item
    Recursive functions make the code look clean and elegant.
  \item
    A complex task can be broken down into simpler sub-problems using
    recursion.
  \item
    Sequence generation is easier with recursion than using some nested
    iteration.
  \end{itemize}
\item
  Disadvantages of recursion

  \begin{itemize}
  \tightlist
  \item
    Sometimes the logic behind recursion is hard to follow through.
  \item
    Recursive calls are expensive (inefficient) as they take up a lot of
    memory and time.
  \item
    Recursive functions are hard to debug.
  \end{itemize}
\end{itemize}

    \textbf{Ex1:} Python Program to Display Fibonacci Sequence Using
Recursion

A Fibonacci sequence is the integer sequence of 0, 1, 1, 2, 3, 5, 8....

\begin{Shaded}
\begin{Highlighting}[]
\KeywordTok{def}\NormalTok{ fibo_recursion(n):}
    \CommentTok{# your code goes here}

\ControlFlowTok{for}\NormalTok{ i }\KeywordTok{in} \BuiltInTok{range}\NormalTok{(}\DecValTok{10}\NormalTok{):}
    \BuiltInTok{print}\NormalTok{ fibo(i), }

\CommentTok{# Output: 0 1 1 2 3 5 8 13 21 34}
\end{Highlighting}
\end{Shaded}

    \textbf{Ex2:} Find Sum of Natural Numbers Using Recursion

\begin{Shaded}
\begin{Highlighting}[]
\KeywordTok{def}\NormalTok{ sum_recursion(n):}
    \CommentTok{# your code goes here}
    
\BuiltInTok{print}\NormalTok{ sum_recursion(}\DecValTok{10}\NormalTok{) }\CommentTok{# Sum of 1 + 2 + 3 + ... + 10}

\CommentTok{# Output: 55}
\end{Highlighting}
\end{Shaded}

    \section{Python Errors and Built-in
Exceptions}\label{python-errors-and-built-in-exceptions}

When writing a program, we, more often than not, will encounter errors.

Error caused by not following the proper structure (syntax) of the
language is called syntax error or parsing error.

\begin{Shaded}
\begin{Highlighting}[]
\OperatorTok{>>>} \ControlFlowTok{if}\NormalTok{ a }\OperatorTok{<} \DecValTok{3}
\NormalTok{  File }\StringTok{"<interactive input>"}\NormalTok{, line }\DecValTok{1}
    \ControlFlowTok{if}\NormalTok{ a }\OperatorTok{<} \DecValTok{3}
           \OperatorTok{^}
\PreprocessorTok{SyntaxError}\NormalTok{: invalid syntax}
\end{Highlighting}
\end{Shaded}

We can notice here that a colon is missing in the if statement.

Errors can also occur at runtime and these are called exceptions. They
occur, for example, when a file we try to open does not exist
(\texttt{FileNotFoundError}), dividing a number by zero
(\texttt{ZeroDivisionError}), module we try to import is not found
(\texttt{ImportError}) etc.

Different kinds of errors can occur in a program, and it is useful to
distinguish among them in order to track them down more quickly:

\begin{itemize}
\tightlist
\item
  Syntax errors are produced by Python when it is translating the source
  code into byte code. They usually indicate that there is something
  wrong with the syntax of the program. Example: Omitting the colon at
  the end of a def statement yields the somewhat redundant message
  SyntaxError: invalid syntax.
\item
  Runtime errors are produced by the runtime system if something goes
  wrong while the program is running. Most runtime error messages
  include information about where the error occurred and what functions
  were executing. Example: An infinite recursion eventually causes a
  runtime error of "maximum recursion depth exceeded."
\item
  Semantic errors are problems with a program that compiles and runs but
  doesn't do the right thing. Example: An expression may not be
  evaluated in the order you expect, yielding an unexpected result.
\end{itemize}

The first step in debugging is to figure out which kind of error you are
dealing with. Although the following sections are organized by error
type, some techniques are applicable in more than one situation.

    Python Built-in Exceptions:

\begin{longtable}[]{@{}ll@{}}
\toprule
\begin{minipage}[b]{0.15\columnwidth}\raggedright\strut
Exception\strut
\end{minipage} & \begin{minipage}[b]{0.79\columnwidth}\raggedright\strut
Cause of Error\strut
\end{minipage}\tabularnewline
\midrule
\endhead
\begin{minipage}[t]{0.15\columnwidth}\raggedright\strut
AssertionError\strut
\end{minipage} & \begin{minipage}[t]{0.79\columnwidth}\raggedright\strut
Raised when assert statement fails.\strut
\end{minipage}\tabularnewline
\begin{minipage}[t]{0.15\columnwidth}\raggedright\strut
AttributeError\strut
\end{minipage} & \begin{minipage}[t]{0.79\columnwidth}\raggedright\strut
Raised when attribute assignment or reference fails.\strut
\end{minipage}\tabularnewline
\begin{minipage}[t]{0.15\columnwidth}\raggedright\strut
EOFError\strut
\end{minipage} & \begin{minipage}[t]{0.79\columnwidth}\raggedright\strut
Raised when the input() functions hits end-of-file condition.\strut
\end{minipage}\tabularnewline
\begin{minipage}[t]{0.15\columnwidth}\raggedright\strut
FloatingPointError\strut
\end{minipage} & \begin{minipage}[t]{0.79\columnwidth}\raggedright\strut
Raised when a floating point operation fails.\strut
\end{minipage}\tabularnewline
\begin{minipage}[t]{0.15\columnwidth}\raggedright\strut
GeneratorExit\strut
\end{minipage} & \begin{minipage}[t]{0.79\columnwidth}\raggedright\strut
Raise when a generator's close() method is called.\strut
\end{minipage}\tabularnewline
\begin{minipage}[t]{0.15\columnwidth}\raggedright\strut
ImportError\strut
\end{minipage} & \begin{minipage}[t]{0.79\columnwidth}\raggedright\strut
Raised when the imported module is not found.\strut
\end{minipage}\tabularnewline
\begin{minipage}[t]{0.15\columnwidth}\raggedright\strut
IndexError\strut
\end{minipage} & \begin{minipage}[t]{0.79\columnwidth}\raggedright\strut
Raised when index of a sequence is out of range.\strut
\end{minipage}\tabularnewline
\begin{minipage}[t]{0.15\columnwidth}\raggedright\strut
KeyError\strut
\end{minipage} & \begin{minipage}[t]{0.79\columnwidth}\raggedright\strut
Raised when a key is not found in a dictionary.\strut
\end{minipage}\tabularnewline
\begin{minipage}[t]{0.15\columnwidth}\raggedright\strut
KeyboardInterrupt\strut
\end{minipage} & \begin{minipage}[t]{0.79\columnwidth}\raggedright\strut
Raised when the user hits interrupt key (Ctrl+c or delete).\strut
\end{minipage}\tabularnewline
\begin{minipage}[t]{0.15\columnwidth}\raggedright\strut
MemoryError\strut
\end{minipage} & \begin{minipage}[t]{0.79\columnwidth}\raggedright\strut
Raised when an operation runs out of memory.\strut
\end{minipage}\tabularnewline
\begin{minipage}[t]{0.15\columnwidth}\raggedright\strut
NameError\strut
\end{minipage} & \begin{minipage}[t]{0.79\columnwidth}\raggedright\strut
Raised when a variable is not found in local or global scope.\strut
\end{minipage}\tabularnewline
\begin{minipage}[t]{0.15\columnwidth}\raggedright\strut
NotImplementedError\strut
\end{minipage} & \begin{minipage}[t]{0.79\columnwidth}\raggedright\strut
Raised by abstract methods.\strut
\end{minipage}\tabularnewline
\begin{minipage}[t]{0.15\columnwidth}\raggedright\strut
OSError\strut
\end{minipage} & \begin{minipage}[t]{0.79\columnwidth}\raggedright\strut
Raised when system operation causes system related error.\strut
\end{minipage}\tabularnewline
\begin{minipage}[t]{0.15\columnwidth}\raggedright\strut
OverflowError\strut
\end{minipage} & \begin{minipage}[t]{0.79\columnwidth}\raggedright\strut
Raised when result of an arithmetic operation is too large to be
represented.\strut
\end{minipage}\tabularnewline
\begin{minipage}[t]{0.15\columnwidth}\raggedright\strut
ReferenceError\strut
\end{minipage} & \begin{minipage}[t]{0.79\columnwidth}\raggedright\strut
Raised when a weak reference proxy is used to access a garbage collected
referent.\strut
\end{minipage}\tabularnewline
\begin{minipage}[t]{0.15\columnwidth}\raggedright\strut
RuntimeError\strut
\end{minipage} & \begin{minipage}[t]{0.79\columnwidth}\raggedright\strut
Raised when an error does not fall under any other category.\strut
\end{minipage}\tabularnewline
\begin{minipage}[t]{0.15\columnwidth}\raggedright\strut
StopIteration\strut
\end{minipage} & \begin{minipage}[t]{0.79\columnwidth}\raggedright\strut
Raised by next() function to indicate that there is no further item to
be returned by iterator.\strut
\end{minipage}\tabularnewline
\begin{minipage}[t]{0.15\columnwidth}\raggedright\strut
SyntaxError\strut
\end{minipage} & \begin{minipage}[t]{0.79\columnwidth}\raggedright\strut
Raised by parser when syntax error is encountered.\strut
\end{minipage}\tabularnewline
\begin{minipage}[t]{0.15\columnwidth}\raggedright\strut
IndentationError\strut
\end{minipage} & \begin{minipage}[t]{0.79\columnwidth}\raggedright\strut
Raised when there is incorrect indentation.\strut
\end{minipage}\tabularnewline
\begin{minipage}[t]{0.15\columnwidth}\raggedright\strut
TabError\strut
\end{minipage} & \begin{minipage}[t]{0.79\columnwidth}\raggedright\strut
Raised when indentation consists of inconsistent tabs and spaces.\strut
\end{minipage}\tabularnewline
\begin{minipage}[t]{0.15\columnwidth}\raggedright\strut
SystemError\strut
\end{minipage} & \begin{minipage}[t]{0.79\columnwidth}\raggedright\strut
Raised when interpreter detects internal error.\strut
\end{minipage}\tabularnewline
\begin{minipage}[t]{0.15\columnwidth}\raggedright\strut
SystemExit\strut
\end{minipage} & \begin{minipage}[t]{0.79\columnwidth}\raggedright\strut
Raised by sys.exit() function.\strut
\end{minipage}\tabularnewline
\begin{minipage}[t]{0.15\columnwidth}\raggedright\strut
TypeError\strut
\end{minipage} & \begin{minipage}[t]{0.79\columnwidth}\raggedright\strut
Raised when a function or operation is applied to an object of incorrect
type.\strut
\end{minipage}\tabularnewline
\begin{minipage}[t]{0.15\columnwidth}\raggedright\strut
UnboundLocalError\strut
\end{minipage} & \begin{minipage}[t]{0.79\columnwidth}\raggedright\strut
Raised when a reference is made to a local variable in a function or
method, but no value has been bound to that variable.\strut
\end{minipage}\tabularnewline
\begin{minipage}[t]{0.15\columnwidth}\raggedright\strut
UnicodeError\strut
\end{minipage} & \begin{minipage}[t]{0.79\columnwidth}\raggedright\strut
Raised when a Unicode-related encoding or decoding error occurs.\strut
\end{minipage}\tabularnewline
\begin{minipage}[t]{0.15\columnwidth}\raggedright\strut
UnicodeEncodeError\strut
\end{minipage} & \begin{minipage}[t]{0.79\columnwidth}\raggedright\strut
Raised when a Unicode-related error occurs during encoding.\strut
\end{minipage}\tabularnewline
\begin{minipage}[t]{0.15\columnwidth}\raggedright\strut
UnicodeDecodeError\strut
\end{minipage} & \begin{minipage}[t]{0.79\columnwidth}\raggedright\strut
Raised when a Unicode-related error occurs during decoding.\strut
\end{minipage}\tabularnewline
\begin{minipage}[t]{0.15\columnwidth}\raggedright\strut
UnicodeTranslateError\strut
\end{minipage} & \begin{minipage}[t]{0.79\columnwidth}\raggedright\strut
Raised when a Unicode-related error occurs during translating.\strut
\end{minipage}\tabularnewline
\begin{minipage}[t]{0.15\columnwidth}\raggedright\strut
ValueError\strut
\end{minipage} & \begin{minipage}[t]{0.79\columnwidth}\raggedright\strut
Raised when a function gets argument of correct type but improper
value.\strut
\end{minipage}\tabularnewline
\begin{minipage}[t]{0.15\columnwidth}\raggedright\strut
ZeroDivisionError\strut
\end{minipage} & \begin{minipage}[t]{0.79\columnwidth}\raggedright\strut
Raised when second operand of division or modulo operation is
zero.\strut
\end{minipage}\tabularnewline
\bottomrule
\end{longtable}

    \subsection{Syntax errors}\label{syntax-errors}

Syntax errors are usually easy to fix once you figure out what they are.
Unfortunately, the error messages are often not helpful. The most common
messages are \textbf{SyntaxError: invalid syntax} and
\textbf{SyntaxError: invalid token}, neither of which is very
informative.

On the other hand, the message does tell you where in the program the
problem occurred. Actually, it tells you where Python noticed a problem,
which is not necessarily where the error is. Sometimes the error is
prior to the location of the error message, often on the preceding line.

If you are building the program incrementally, you should have a good
idea about where the error is. It will be in the last line you added.

If you are copying code from a book, start by comparing your code to the
book's code very carefully. Check every character. At the same time,
remember that the book might be wrong, so if you see something that
looks like a syntax error, it might be.

Here are some ways to avoid the most common syntax errors:

\begin{itemize}
\tightlist
\item
  Make sure you are not using a Python keyword for a variable name.
\item
  Check that you have a colon at the end of the header of every compound
  statement, including for, while, if, and def statements.
\item
  Check that indentation is consistent. You may indent with either
  spaces or tabs but it's best not to mix them. Each level should be
  nested the same amount.
\item
  Make sure that any strings in the code have matching quotation marks.
\item
  If you have multiline strings with triple quotes (single or double),
  make sure you have terminated the string properly. An unterminated
  string may cause an invalid token error at the end of your program, or
  it may treat the following part of the program as a string until it
  comes to the next string. In the second case, it might not produce an
  error message at all!
\item
  An unclosed bracket (, \{, or {[} makes Python continue with the next
  line as part of the current statement. Generally, an error occurs
  almost immediately in the next line.
\item
  Check for the classic = instead of == inside a conditional.
\end{itemize}

If nothing works, move on to the next section...

    \subsection{Runtime errors}\label{runtime-errors}

Once your program is syntactically correct, Python can import it and at
least start running it. What could possibly go wrong?

\subsubsection{My program hangs.}\label{my-program-hangs.}

If a program stops and seems to be doing nothing, we say it is
"hanging." Often that means that it is caught in an infinite loop or an
infinite recursion.

\begin{itemize}
\tightlist
\item
  If there is a particular loop that you suspect is the problem, add a
  print statement immediately before the loop that says "entering the
  loop" and another immediately after that says "exiting the loop."
\item
  Run the program. If you get the first message and not the second,
  you've got an infinite loop. Go to the "Infinite Loop" section below.
\item
  Most of the time, an infinite recursion will cause the program to run
  for a while and then produce a "RuntimeError: Maximum recursion depth
  exceeded" error. If that happens, go to the "Infinite Recursion"
  section below.
\item
  If you are not getting this error but you suspect there is a problem
  with a recursive method or function, you can still use the techniques
  in the "Infinite Recursion" section.
\item
  If neither of those steps works, start testing other loops and other
  recursive functions and methods.
\end{itemize}

    \subsubsection{Infinite Loop}\label{infinite-loop}

If you think you have an infinite loop and you think you know what loop
is causing the problem, add a \texttt{print} statement at the end of the
loop that prints the values of the variables in the condition and the
value of the condition.

\begin{Shaded}
\begin{Highlighting}[]
\ControlFlowTok{while}\NormalTok{ x }\OperatorTok{>} \DecValTok{0} \KeywordTok{and}\NormalTok{ y }\OperatorTok{<} \DecValTok{0}\NormalTok{ : }
  \CommentTok{# do something to x }
  \CommentTok{# do something to y }

  \BuiltInTok{print}  \StringTok{"x: "}\NormalTok{, x }
  \BuiltInTok{print}  \StringTok{"y: "}\NormalTok{, y }
  \BuiltInTok{print}  \StringTok{"condition: "}\NormalTok{, (x }\OperatorTok{>} \DecValTok{0} \KeywordTok{and}\NormalTok{ y }\OperatorTok{<} \DecValTok{0}\NormalTok{) }
\end{Highlighting}
\end{Shaded}

Now when you run the program, you will see three lines of output for
each time through the loop. The last time through the loop, the
condition should be false. If the loop keeps going, you will be able to
see the values of x and y, and you might figure out why they are not
being updated correctly.

\subsubsection{Infinite Recursion}\label{infinite-recursion}

Most of the time, an infinite recursion will cause the program to run
for a while and then produce a Maximum recursion depth exceeded error.

\subsubsection{Flow of Execution}\label{flow-of-execution}

If you are not sure how the flow of execution is moving through your
program, add print statements to the beginning of each function with a
message like "entering function foo," where foo is the name of the
function.

Now when you run the program, it will print a trace of each function as
it is invoked.

    \section{References}\label{references}

\begin{itemize}
\tightlist
\item
  http://www.greenteapress.com/thinkpython/thinkCSpy/html/app01.html
\item
  http://www.python-course.eu/python3\_lambda.php
\end{itemize}


    % Add a bibliography block to the postdoc
    
    
    
    \end{document}
